%%%%%%%%%%%%%%%%%%%%%%%%%%%%%%%%%%%%%%%%%%%%%%%%%%%%%%%%%%%%%%%%%%%%%%%%
%%% HMT/YGL/GEB QB Approx table

\begin{table}
\begin{center}
\begin{tabular}{|c|c|c|c|c|c|c|c|c|}
\cline{2-9}
\multicolumn{1}{c|}{} &
\multicolumn{2}{|c|}{\textbf{1E-14}} & \multicolumn{2}{|c|}{\textbf{5E-15}} &
\multicolumn{2}{|c|}{\textbf{1E-15}} & \multicolumn{2}{|c|}{\textbf{5E-16}} \\
\hline
Matrix E1 &
Err & Samp & Err & Samp &
Err & Samp & Err & Samp \\
\hline
GEB & 1E-14 & 97 & 4E-15 & 100 & 6E-16  & 144* & 7E-16 & 200+ \\
\hline
\hline
\multicolumn{1}{c|}{} &
\multicolumn{2}{|c|}{\textbf{5E-13}} & \multicolumn{2}{|c|}{\textbf{1E-13}} &
\multicolumn{2}{|c|}{\textbf{4E-14}} & \multicolumn{2}{|c|}{\textbf{1E-14}} \\
\hline
Matrix S &
Err & Samp & Err & Samp &
Err & Samp & Err & Samp \\
\hline
GEB & 4E-13 & 75 & 7E-14 & 96 & 5E-14 & 100 & 9E-15 & 107 \\
\hline
\hline
\multicolumn{1}{c|}{} &
\multicolumn{2}{|c|}{\textbf{5E-5}} & \multicolumn{2}{|c|}{\textbf{1E-5}} &
\multicolumn{2}{|c|}{\textbf{5E-6}} & \multicolumn{2}{|c|}{\textbf{1E-6}} \\
\hline
Matrix D &
Err & Samp & Err & Samp &
Err & Samp & Err & Samp \\
\hline
GEB (s) & 4E-5 & 59 & 9E-6 & 66 & 9E-6 & 71 & 6E-7 & 186* \\
\hline
\end{tabular}
\end{center}
\caption[QB Adaptive Approximation Results (Stringent GEB Tests)]{
Tough QB approximation results for Matrices E1, S, and D
using the new GEB stopping criterion.
For each absolute error tolerance, we averaged 1,000 trials to determine
the average error (Err) and average samples used (Samp) in order to compute
a QB approximation once we used the new stopping criterion to
determine when we had approximated the range.
Random samples were computed in blocks of 5.
For matrix D, we used \textbf{single precision}.
In the case when we used $200+$ samples,
we were not able to meet the HMT stopping criterion
and used the maximum of 200 random samples.
The minimum possible ranks can be found in
Table~\ref{tab:results_qb_min_rank_tough_geb}.
A ``*'' means that there were some runs which reached the maximum allowable
samples of 200 before compression.
For matrix E1 with $\epsr$=1E-15, 7.7\% of the trials used over 200 samples
(and so failed to stop);
for matrix D  with $\epsr$=1F-6, 73.8\% of the trials used over 200 samples.
}
\label{tab:results_qb_approx_mat_geb_hard}
\end{table}

\section{Proof of 1D Birkhoff Interpolation for polynomials of degree $2n$
    with Endpoint Interpolation}
\label{sec:cvip_birkhoff_bdy}

The section here is similar to Sec.~\ref{sec:cvip_birkhoff_pt},
except that in this case we interpolate function values
at the endpoints, giving us additional information $f(-1)$ and $f(1)$.
After a set of Givens rotations, the same from the previous
section, we see
%
\begin{equation}
    \setcounter{MaxMatrixCols}{20}
    \begin{bmatrix}
        0 & \gamma_{2} & \\
        & & \gamma_{3} & \\
        & & & \ddots & \\
        & & & & \gamma_{n-1} \\
        & & & & & \gamma_{n} \\
        & & & & & & \gamma_{n+1} \\
        & \alpha_{2} & & & & \alpha_{n} & & \beta_{n} & &
            \cdots & & \beta_{2} & \\
        \alpha_{1} & & \alpha_{3} & & \alpha_{n-1} & & \alpha_{n+1} &
            & \beta_{n-1} & & \beta_{3} & & \beta_{1} \\
    \end{bmatrix},
\end{equation}
%
where we are assuming $n$ is even; similar results hold when $n$ is odd.
In this case, we have
%
\begin{align}
    \gamma_{k} &= \frac{k-1}{c_{k}k^{s}} \qquad k\in\braces{2,\cdots,n}
        \nonumber\\
    \gamma_{n+1} &= \frac{n}{\parens{n+1}^{s}} \nonumber\\
    \alpha_{1} &= 1^{-s} \nonumber\\
    \alpha_{k} &= c_{k}k^{-s}\brackets{1 +
        \parens{\frac{k-1}{2n+1-k}}\tau_{k}^{2}} \qquad k\in\braces{2,\cdots,n}
            \nonumber\\
    \alpha_{n+1} &= \parens{n+1}^{-s} \nonumber\\
    \beta_{1} &= \parens{2n+1}^{-s} \nonumber\\
    \beta_{k} &= 2\parens{\frac{n+1-k}{2n+1-k}}s_{k}k^{-s}
        \qquad k\in\braces{2,\cdots,n}.
\end{align}
%
It follows
%
\begin{align}
    \gamma_{k}^{-1} &\le \frac{k^{s}}{k-1} \nonumber\\
        &\le 2^{s}\parens{k-1}^{s-1} \qquad k\in\braces{2,\cdots,n+1}.
    \label{eq:birkhoff_bdy_gamma_bound}
\end{align}
%
Both inequalities hold but are used in different places.
When $s\ge1$, we have $c_{k}\ge\frac{1}{\sqrt{5}}$ and
%
\begin{align}
    \abs{\beta_{k}} &\ge \frac{2}{\sqrt{5}n}\parens{2n+2-k}^{-s} \qquad
        k\in\braces{1,\cdots,n} \nonumber\\
    \abs{\beta_{k}} &\le 2n\parens{2n+2-k}^{-s} \qquad
        k\in\braces{1,\cdots,n} \nonumber\\
    \alpha_{k} &\le 3k^{-s} \qquad k\in\braces{2,\cdots,n+1}.
    \label{eq:birkhoff_bdy_alpha_beta_bounds}
\end{align}



After a column permutation and a pair of Householder reflectors we have
%
\begin{equation}
    \setcounter{MaxMatrixCols}{20}
    \begin{bmatrix}
        \gamma_{2} & \\
        & \gamma_{3} & \\
        & & \ddots & \\
        & & & \gamma_{n-1} \\
        & & & & \gamma_{n} \\
        & & & & & \gamma_{n+1} \\
        \alpha_{2} & & \cdots & & \alpha_{n} & & \ell_{1} \\
        & \alpha_{3} & & \alpha_{n-1} & & \alpha_{n+1} &
            & \ell_{2} \\
    \end{bmatrix}
        = \begin{bmatrix} \Gamma & \\ \widehat{A} & \widehat{L} \end{bmatrix},
\end{equation}
%
where
%
\begin{align}
    \ell_{1}^{2} &= \beta_{2}^{2} + \beta_{4}^{2} + \cdots + \beta_{n}^{2}
        \nonumber\\
    \ell_{2}^{2} &= \beta_{1}^{2} + \beta_{3}^{2} + \cdots + \beta_{n-1}^{2}
            + \alpha_{1}^{2}.
\end{align}
%
Thus, we have finally arrived at our $LQ$ factorization:
%
\begin{equation}
    \begin{bmatrix} UWD \\ A \end{bmatrix} = 
        \begin{bmatrix} \Gamma & & \\ \widehat{A} & \widehat{L} & 0
        \end{bmatrix} H\Pi G,
\end{equation}
%
where $G$ is our set of Givens rotations, $\Pi$ is our permutation matrix,
and $H$ is our Householder reflector's matrix.
$\Pi$ and $H$ satisfy
%
\begin{equation}
    \Pi^{*} \begin{bmatrix} x_{1} \\ x_{2} \\ \vdots \\ x_{2n} \\ x_{2n+1}
        \end{bmatrix}
    = \begin{bmatrix} x_{2n+1} \\ x_{1} \\ \vdots \\ x_{2n-1} \\ x_{2n}
        \end{bmatrix}
\end{equation}
%
and
%
\begin{equation}
    H = \begin{bmatrix} I_{n} & \\ & I_{n+1} - 2VT^{-1}V^{*} \end{bmatrix}
\end{equation}
%
with
%
\begin{align}
    V^{*} &= \begin{bmatrix}
        \beta_{n} & & \cdots & & \beta_{2} & & \\
        & \beta_{n-1} & & \beta_{3} & & \beta_{1} & \alpha_{1} \\
    \end{bmatrix} \nonumber\\
    T &= \begin{bmatrix} \ell_{1}^{2} & \\ & \ell_{2}^{2} \end{bmatrix}.
\end{align}

We have
%
\begin{equation}
\begin{bmatrix} \Gamma & \\ \widehat{A} & \widehat{L}
        \end{bmatrix}^{-1} \begin{bmatrix} \hat{f}' \\ \tilde{f} \end{bmatrix}
    = \begin{bmatrix} \Gamma^{-1}\hat{f}' \\
        \widehat{L}^{-1}(\tilde{f}-\widehat{A}\Gamma^{-1}\hat{f}') \end{bmatrix}
\end{equation}
%
and, following a similar pattern from Sec.~\ref{sec:cvip_interp_1D_bdy}, set
%
\begin{align}
    \hat{g} &= \tilde{f}-\widehat{A}\Gamma^{-1}\hat{f}' \nonumber\\
    \hat{h} &= \widehat{L}^{-1}\hat{g}.
\end{align}
%
At this point, it is clear we have the MSN solution is
%
\begin{align}
    \tilde{a} &= D_{s}^{-1}G^{*}\Pi^{*}H^{*}
        \begin{bmatrix} \Gamma^{-1}\hat{f}' \\ \hat{h} \\ 0 \end{bmatrix}
        \nonumber\\
    &= \parens{
        D_{s}^{-1}G^{*} \begin{bmatrix} 0 \\ \Gamma^{-1}\hat{f}' \\ 0
        \end{bmatrix} + \begin{bmatrix} a_{0} \\ 0 \\ 0 \end{bmatrix}}
    + \parens{
        D_{s}^{-1}G^{*}\Pi^{*}H^{*} \begin{bmatrix} 0 \\ \hat{h} \\ 0
        \end{bmatrix} - \begin{bmatrix} a_{0} \\ 0 \\ 0 \end{bmatrix}}
        \nonumber\\
    &= \tilde{a}_{1} + \tilde{a}_{2}.
\end{align}
%
Here, we are adding and subtracting the first true coefficient $a_{0}$
of the Chebyshev series. Thus, $\tilde{a}_{1}$ is $\tilde{a}$ from
Sec.~\ref{sec:cvip_birkhoff_pt}.
We will compute
%
\begin{equation}
    \norm{a_{c} - \tilde{a}}_{p,1} \le \norm{a_{c}-\tilde{a}_{1}}_{p,1} + 
        \norm{\tilde{a}_{2}}_{p,1}
\end{equation}
%
and focus on bounding $\norm{\tilde{a}_{2}}_{p,1}$.

First, we will need to compute bounds on $\ell_{1}$ and $\ell_{2}$.
We find
%
\begin{align}
    \ell_{1}^{2} &= \beta_{2}^{2} + \beta_{4}^{2} + \cdots + \beta_{n}^{2}
        \nonumber\\
    &\ge \frac{4}{5n^{2}} \sum_{k=1}^{\frac{n}{2}} \brackets{n + 2k}^{-2s}
        \nonumber\\
    &\ge \frac{1}{5sn^{2s+1}}.
\end{align}
%
Similarly, we have
%
\begin{align}
    \ell_{1}^{2} &= \beta_{1}^{2} + \beta_{3}^{2} + \cdots + \beta_{n-1}^{2}
        + \alpha_{1}^{2}
        \nonumber\\
    &\ge 1 + \frac{4}{5n^{2}} \sum_{k=1}^{\frac{n}{2}} \brackets{n + 2k}^{-2s}
        \nonumber\\
    &\ge 1 + \frac{1}{5sn^{2s+1}}.
\end{align}
%
Clearly, we have
%
\begin{equation}
    \ell_{1}^{-2},\ell_{2}^{-2} \le 5sn^{2s+1},
\end{equation}
%
but we will need the tighter bound of $\ell_{1}^{-1}\le1$.

We can now use these bounds to look at $\tilde{a}_{2}$.
First, we see
%
\begin{align}
    \alpha_{k}\gamma_{k}^{-1}\hat{f}_{k}'
        &= c_{k}^{2}a_{k-1} + \frac{c_{k}^{2}}{k-1}\eta_{k-1}
        + s_{k}^{2}\parens{\frac{k-1}{2n+1-k}}a_{k-1}
        + \frac{s_{k}^{2}}{2n+1-k}\eta_{k-1} \nonumber\\
    &\qquad k\in\braces{2,\cdots,n} \nonumber\\
    \alpha_{n+1}\gamma_{n+1}^{-1}\hat{f}_{n+1}' &= a_{n} + \frac{1}{n}\eta_{n}.
\end{align}
%
Thus, it follows
%
\begin{align}
    a_{k-1} - \alpha_{k}\gamma_{k}^{-1}\hat{f}_{k}' &=
        2s_{k}^{2}\parens{\frac{n+1-k}{2n+1-k}}a_{k-1}
        - \brackets{\frac{c_{k}^{2}}{k-1} + \frac{s_{k}^{2}}{2n+1-k}}\eta_{k-1}
        \nonumber\\
    &\qquad k\in\braces{2,\cdots,n} \nonumber\\
    a_{n} - \alpha_{n+1}\gamma_{n+1}^{-1}\hat{f}_{n+1}' &=
        -\frac{1}{n}\eta_{n}.
\end{align}
%
Thus, we see
%
\begin{align}
    \hat{g}_{1} &= 2\sum_{\substack{k=2\\k\text{ even}}}^{n}
        s_{k}^{2}\parens{\frac{n+1-k}{2n+1-k}}a_{k-1}
        - \sum_{\substack{k=2\\k\text{ even}}}^{n}
        \brackets{\frac{c_{k}^{2}}{k-1} + \frac{s_{k}^{2}}{2n+1-k}}\eta_{k-1}
        \nonumber\\
    \hat{g}_{2} &= a_{0} + 2\sum_{\substack{k=2\\k\text{ odd}}}^{n}
        s_{k}^{2}\parens{\frac{n+1-k}{2n+1-k}}a_{k-1}
        - \frac{1}{n}\eta_{n} 
        - \sum_{\substack{k=2\\k\text{ odd}}}^{n} 
        \brackets{\frac{c_{k}^{2}}{k-1} + \frac{s_{k}^{2}}{2n+1-k}}\eta_{k-1}
\end{align}
%
This implies
%
\begin{equation}
    \abs{\hat{g}_{1}} + \abs{\hat{g}_{2} - a_{0}} \le
        \sum_{k=2}^{n}s_{k}^{2}\abs{a_{k-1}} + \sum_{k=1}^{n}\abs{\eta_{k}}.
\end{equation}
%
The reason for this bound will be clear shortly. Even so,
because
%
\begin{align}
    \abs{s_{k}} &\le \abs{\tau_{k}} \nonumber\\
        &\le 2\parens{\frac{k}{2n+2-k}}^{s-1},
\end{align}
%
we have the bounds
%
\begin{equation}
    \abs{\hat{g}_{1}} + \abs{\hat{g}_{2} - a_{0}} \le
        4\sqrt{\frac{2}{2s-3}}\frac{\norm{a}_{s}}{n^{s-\frac{1}{2}}}
        + \frac{1}{\sqrt{2s-3}}\frac{\norm{a}_{s}}{n^{s-\frac{3}{2}}}.
\end{equation}

Thus, we see
%
\begin{align}
    \hat{h}_{1} &= \ell_{1}^{-1}\hat{g}_{1} \nonumber\\
    \hat{h}_{2} &= \ell_{2}^{-1}\hat{g}_{2} \nonumber\\
        &= \ell_{2}^{-1}\brackets{\parens{\hat{g}_{2} - a_{0}} + a_{0}}
\end{align}
%
which implies
%
\begin{align}
    |\hat{h}_{1}| &\le C_{s}\norm{a}_{s}n^{2} \nonumber\\
    |\hat{h}_{2}| &\le \frac{C_{s}\norm{a}_{s}}{n^{s-\frac{3}{2}}}
        + \abs{a_{0}} \nonumber\\
\end{align} 

\clearpage

For this section, we interpolate derivative values on the Chebyshev
nodes and function values on the boundary points.
We have the requirements
%
\begin{equation}
    \begin{bmatrix}UWD\\A\end{bmatrix} \parens{a-\tilde{a}} =
        \begin{bmatrix}\hat{b}\\\hat{c}\end{bmatrix}.
\end{equation}
%
Using ideas from Eq.~\eqref{eq:CS_coefs_powers}, we have the bounds
%
\begin{align}
    |\hat{b}_{k}| &\le \sum_{\ell=1}^{\infty}\parens{2\ell n +k}
        \abs{a_{2\ell n+k}} + \sum_{\ell=2}^{\infty}\parens{2\ell n-k}
        \abs{a_{2\ell n-k}} \nonumber\\
    &\le \frac{\norm{a}_{\sigma}}{2^{\sigma-2}n^{\sigma-1}}\sqrt{
        \zeta(2\sigma-2)} \nonumber\\
            &\qquad k\in\braces{1,\cdots,n-1} \nonumber\\
    |\hat{b}_{n}| &\le \sum_{\ell=1}^{\infty} \brackets{(2\ell+1)n+1}
        \abs{a_{(2\ell+1)n+1}} \nonumber\\
    &\le \frac{\norm{a}_{\sigma}}{2^{\sigma-1}n^{\sigma-1}}
        \sqrt{\zeta(2\sigma-2)}
        \nonumber\\
    &\le \frac{\norm{a}_{\sigma}}{2^{\sigma-2}n^{\sigma-1}}
        \sqrt{\zeta(2\sigma-2)}.
    \label{eq:birkhoff_bdy_bhat_def}
\end{align}
%
From Eq.~\eqref{eq:interp_bdy_chat_ubounds}, we have
%
\begin{equation}
    |\hat{c}_{1}|, |\hat{c}_{2}| \le
        \frac{C_{\sigma}}{n^{\sigma-\frac{1}{2}}}\norm{a}_{\sigma}.
\end{equation}

After a set of Givens rotations, we see
%
\begin{equation}
    \setcounter{MaxMatrixCols}{20}
    \begin{bmatrix}
        0 & \gamma_{2} & \\
        & & \gamma_{3} & \\
        & & & \ddots & \\
        & & & & \gamma_{n-1} \\
        & & & & & \gamma_{n} \\
        & & & & & & \gamma_{n+1} \\
        & \alpha_{2} & & & & \alpha_{n} & & \beta_{n} & &
            \cdots & & \beta_{2} & \\
        \alpha_{1} & & \alpha_{3} & & \alpha_{n-1} & & \alpha_{n+1} &
            & \beta_{n-1} & & \beta_{3} & & \beta_{1} \\
    \end{bmatrix},
\end{equation}
%
where we are assuming $n$ is even; similar results hold when $n$ is odd.
In this case, we have
%
\begin{align}
    \gamma_{k} &= \parens{k-1}k^{-s}\sqrt{1 + \parens{\frac{2n+1-k}{k-1}}^{2}
            \parens{\frac{k}{2n+2-k}}^{2s}} \nonumber\\
        &\qquad k\in\braces{2,\cdots,n} \nonumber\\
    \gamma_{n+1} &= n\parens{n+1}^{-s} \nonumber\\
    c_{k} &= \braces{1 + \parens{\frac{2n+1-k}{k-1}}^{2}\parens{
        \frac{k}{2n+2-k}}^{2s}}^{-1/2} \nonumber\\
    \alpha_{1} &= 1^{-s} \nonumber\\
    \alpha_{k} &= c_{k}k^{-s}\brackets{1 + \parens{\frac{2n+1-k}{k-1}}
        \parens{\frac{k}{2n+2-k}}^{2s}} \qquad k\in\braces{2,\cdots,n}
            \nonumber\\
    \alpha_{n+1} &= \parens{n+1}^{-s} \nonumber\\
    \beta_{1} &= \parens{2n+1}^{-s} \nonumber\\
    \beta_{k} &= -2c_{k}\parens{2n+2-k}^{-s}\parens{\frac{n+1-k}{k-1}}
        \qquad k\in\braces{2,\cdots,n}.
\end{align}
%
The $c_{k}$ in the previous equation are the cosine terms from the
Givens rotations; they should not be confused with
$\hat{c}_{1}$ and $\hat{c}_{2}$.
From here, it follows
%
\begin{align}
    \gamma_{k}^{-1} &\le \frac{k^{s}}{k-1} \nonumber\\
        &\le 2^{s}\parens{k-1}^{s-1} \qquad k\in\braces{2,\cdots,n+1}.
    \label{eq:birkhoff_bdy_gamma_bound}
\end{align}
%
Both inequalities hold but are used in different places.
When $s\ge1$, we have $c_{k}\ge\frac{1}{\sqrt{5}}$ and
%
\begin{align}
    \abs{\beta_{k}} &\ge \frac{2}{\sqrt{5}n}\parens{2n+2-k}^{-s} \qquad
        k\in\braces{1,\cdots,n} \nonumber\\
    \abs{\beta_{k}} &\le 2n\parens{2n+2-k}^{-s} \qquad
        k\in\braces{1,\cdots,n} \nonumber\\
    \alpha_{k} &\le 3k^{-s} \qquad k\in\braces{2,\cdots,n+1}.
    \label{eq:birkhoff_bdy_alpha_beta_bounds}
\end{align}



After a column permutation and a pair of Householder reflectors we have
%
\begin{equation}
    \setcounter{MaxMatrixCols}{20}
    \begin{bmatrix}
        \gamma_{2} & \\
        & \gamma_{3} & \\
        & & \ddots & \\
        & & & \gamma_{n-1} \\
        & & & & \gamma_{n} \\
        & & & & & \gamma_{n+1} \\
        \alpha_{2} & & \cdots & & \alpha_{n} & & \ell_{1} \\
        & \alpha_{3} & & \alpha_{n-1} & & \alpha_{n+1} &
            & \ell_{2} \\
    \end{bmatrix}
        = \begin{bmatrix} \Gamma & \\ \widehat{A} & \widehat{L} \end{bmatrix},
\end{equation}
%
where
%
\begin{align}
    \ell_{1}^{2} &= \beta_{2}^{2} + \beta_{4}^{2} + \cdots + \beta_{n}^{2}
        \nonumber\\
    \ell_{2}^{2} &= \beta_{1}^{2} + \beta_{3}^{2} + \cdots + \beta_{n-1}^{2}
            + \alpha_{1}^{2}.
\end{align}

We now focus our attention on computing lower bounds for $\ell_{1}$
and $\ell_{2}$. Using the bounds on $\beta_{k}$ in
Eq.~\eqref{eq:birkhoff_bdy_alpha_beta_bounds}, we see
%
\begin{align}
    \ell_{1}^{2} &= \beta_{2}^{2} + \beta_{4}^{2} + \cdots + \beta_{n}^{2}
        \nonumber\\
    &\ge \frac{4}{5n^{2}}\sum_{k=1}^{\frac{n}{2}}\brackets{n+2k}^{-2s}
        \nonumber\\
    &\ge \frac{1}{25s}\frac{1}{n^{2s+1}},
\end{align}
%
where the last inequality follows from Eq.~\eqref{eq:bound_xp_12}.
Because $\alpha_{1}=1$, $\ell_{2}^{2}\ge1$, and we have
%
\begin{equation}
    \ell_{1}^{-1},\ell_{2}^{-1} \le 5\sqrt{s}n^{s+\frac{1}{2}}.
\end{equation}
%
Furthermore, we have the upper bound
%
\begin{align}
    \ell_{1}^{2},\ell_{2}^{2}
        &\le \alpha_{1}^{2} + \beta_{1}^{2} + \beta_{2}^{2}
            + \cdots + \beta_{n}^{2} \nonumber\\
        &\le 1 + 4n^{2}\sum_{k=1}^{n+1}\brackets{n+k}^{-2s} \nonumber\\
        &\le \begin{cases} 9n &s\ge1 \\ 9 &s\ge\frac{3}{2} \end{cases},
\end{align}
%
so that
%
\begin{equation}
    \ell_{1},\ell_{2} \le \begin{cases}
            3\sqrt{n} &s\ge1 \\ 3 &s\ge\frac{3}{2} \end{cases}.
\end{equation}

We need to compute
%
\begin{equation}
    \norm{\begin{bmatrix} \Gamma &  \\ \widehat{A} & \widehat{L}
        \end{bmatrix}^{-1}\begin{bmatrix} \hat{b} \\ \hat{c}
        \end{bmatrix}}_{2}^{2}
    = ||\Gamma^{-1}\hat{b}||_{2}^{2} + ||\widehat{L}^{-1}
        (\hat{c} - \widehat{A}\Gamma^{-1}\hat{b})||_{2}^{2}.
\end{equation}
%
Because
%
\begin{equation}
    \gamma_{k}^{-1}\hat{b}_{k} \le \frac{\norm{a}_{\sigma}
            \sqrt{\zeta(2\sigma-2)}}{2^{\sigma-s-2}n^{\sigma-1}}
        \parens{k-1}^{s-1}
\end{equation}
%
we have
%
\begin{align}
    ||\Gamma^{-1}\hat{b}||_{2}^{2}
        &\le \brackets{\frac{\norm{a}_{\sigma}\sqrt{\zeta(2\sigma-2)}}{
            2^{\sigma-s-2}n^{\sigma-1}}}^{2}
        \sum_{k=1}^{n}k^{2s-2} \nonumber\\
    &\le \brackets{\frac{\norm{a}_{\sigma}\sqrt{\zeta(2\sigma-2)}}{
            2^{\sigma-s-2}n^{\sigma-1}}}^{2}
        8n^{2s-1},
\end{align}
%
where the first inequality uses Eq.~\eqref{eq:birkhoff_bdy_gamma_bound}
and the second inequality follows from Eq.~\eqref{eq:bound_xp_01}.
Therefore, we have
%
\begin{equation}
    ||\Gamma^{-1}\hat{b}||_{2} \le
        \frac{C_{\sigma,s}}{n^{\sigma-s-\frac{1}{2}}}.
\end{equation}



We will now focus on the $||\widehat{L}^{-1}
(\hat{c} - \widehat{A}\Gamma^{-1}\hat{b})||_{2}^{2}$ term.
First, we see
%
\begin{align}
    \alpha_{2}\gamma_{2}^{-1} + \alpha_{4}\gamma_{4}^{-1} + \cdots
            + \alpha_{n}\gamma_{n}^{-1}
        &\le 3\brackets{1 + \frac{1}{3} + \frac{1}{5} + \cdots + \frac{1}{n-1}}
            \nonumber\\
        &\le 6\ln n \nonumber\\
    \alpha_{3}\gamma_{3}^{-1} + \alpha_{5}\gamma_{5}^{-1} + \cdots
            + \alpha_{n+1}\gamma_{n+1}^{-1}
        &\le 3\brackets{\frac{1}{2} + \frac{1}{4} + \frac{1}{6} + \cdots
            + \frac{1}{n}}
            \nonumber\\
        &\le 6\ln n,
\end{align}
%
where we use the bounds from Eqs.~\eqref{eq:bound_xp_01},
\eqref{eq:birkhoff_bdy_gamma_bound}, and 
\eqref{eq:birkhoff_bdy_alpha_beta_bounds}.



Putting this together, we see
%
\begin{equation}
    |\widehat{A}\Gamma^{-1}\hat{b}| \le \frac{3\norm{a}_{\sigma}}{2^{\sigma-3}}
            \sqrt{\zeta(2\sigma-2)}\frac{\ln n}{n^{\sigma-1}}
        \begin{bmatrix}1\\1\end{bmatrix},
\end{equation}
%
where the above inequality holds componentwise.
Furthermore, we have
%
\begin{align}
    |\hat{c} - \widehat{A}\Gamma^{-1}\hat{b}|
        &\le |\hat{c}| + |\widehat{A}\Gamma^{-1}\hat{b}| \nonumber\\
    &\le C_{\sigma,s}\norm{a}_{\sigma}\frac{\ln n}{n^{\sigma-1}}
        \begin{bmatrix}1\\1\end{bmatrix},
\end{align}
%
where the inequalities hold componentwise.
Finally, we see
%
\begin{equation}
    |\widehat{L}^{-1}(\hat{c} - \widehat{A}\Gamma^{-1}\hat{b})|
        \le C_{\sigma,s}\norm{a}_{\sigma}
            \frac{\ln n}{n^{\sigma-s-\frac{3}{2}}}
        \begin{bmatrix}1\\1\end{bmatrix},
\end{equation}
%
where the inequality is componentwise. Normwise convergence easily follows,
and to obtain convergence, we require
$\sigma>\frac{3}{2}$ and $1\le s < \sigma-\frac{3}{2}$.



We now focus on bounding the condition number $\kappa(L)$.
We see
%
\begin{align}
    \norm{\begin{bmatrix} \Gamma & \\ \widehat{A} & \widehat{L} \end{bmatrix}
        \begin{bmatrix}a\\b\end{bmatrix}}_{2}^{2}
        &= \norm{\Gamma a}_{2}^{2} + ||\widehat{A}a + \widehat{L}b||_{2}^{2}
            \nonumber\\
        &\le 2 + \norm{\begin{bmatrix}
            2\braces{1^{-s}+3^{-s}+\cdots+(n-1)^{-s}}+ \ell_{1} \\
            2\braces{2^{-s} + 4^{-s} + \cdots + n^{-s}} + \ell_{2}
            \end{bmatrix}}_{2}^{2} \nonumber\\
        &\le \begin{cases}
            256n &s\ge1 \\
            1024 &s\ge2
        \end{cases}.
\end{align}
%
Therefore, we have $\norm{L}_{2}\le 16\sqrt{n}$ for $s\ge1$
while $\norm{L}_{2}\le32$ for $s\ge\frac{3}{2}$.




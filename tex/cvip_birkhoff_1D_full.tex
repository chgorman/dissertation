\section{Proof of 1D Full Birkhoff Interpolation for polynomials of degree $2n$}
\label{sec:cvip_birkhoff_full_1D}

We begin by computing the error. Now, we know the coefficients are
%
\begin{align}
    \tilde{a} &= D_{s}^{-1}G^{*} \begin{bmatrix}Y&\\A&B&0\end{bmatrix}^{+}
        \begin{bmatrix} \hat{f} \\ E\hat{f}' \end{bmatrix} \nonumber\\
    &= D_{s}^{-1}G^{*}\begin{bmatrix} Y^{-1}\hat{f} \\
        B^{-1}(E\hat{f}' - AY^{-1}\hat{f}) \\ 0\end{bmatrix} \nonumber\\
    &= D_{s}^{-1}G^{*}\begin{bmatrix} Y^{-1}\hat{f} \\ 0 \\ 0 \end{bmatrix}
        + D_{s}^{-1}G^{*} \begin{bmatrix} 0 \\
        B^{-1}(E\hat{f}' - AY^{-1}\hat{f}) \\ 0\end{bmatrix} \nonumber\\
    &= \tilde{a}_{1} + \tilde{a}_{2}.
    \label{eq:cvip_birkhoff_full_msn_total_terms}
\end{align}
%
$Y$ and $G$ are the same as from Sec.~\ref{sec:cvip_interp_1D}.
Additionally, we have
%
\begin{align}
    A &= \begin{bmatrix}
        & & & & 0 \\
        & & & & \alpha_{n} \\
        & & & \iddots & \\
        & & \alpha_{3} & & \\
        0 & \alpha_{2} & & & \\
        \end{bmatrix}
            \nonumber\\
    B &= \diag\parens{\beta_{n+1},\beta_{n},\cdots,\beta_{2}}
\end{align}
%
with
%
\begin{align}
    y_{k} &= \frac{1}{c_{k}k^{s}} \nonumber\\
    \alpha_{k} &= (k-1)k^{-s}c_{k}\braces{1-\parens{\frac{2n+1-k}{k-1}}
        \tau_{k}^{2}}
        \nonumber\\
    \beta_{k} &= 2nk^{-s}s_{k} \quad k\in\braces{2,\cdots,n} \nonumber\\
    \beta_{n+1} &= n\parens{n+1}^{-s}.
\end{align}
%



Now, $\tilde{a}_{1}$ is the exact term we obtain
from Eq.~\eqref{eq:cvip_interp_1D_msn_coefs}. Thus, we focus on
the second term because
%
\begin{equation}
    \norm{a_{c} - \tilde{a}}_{p,1}
        \le \norm{a_{c}-\tilde{a}_{1}}_{p,1} + \norm{\tilde{a}_{2}}_{p,1}.
\end{equation}
%
We set
%
\begin{align}
    q &= \begin{bmatrix}
        B^{-1}(E\hat{f}' - AY^{-1}\hat{f}) \\ 0\end{bmatrix} \nonumber\\
    &= \begin{bmatrix} q_{n+1} & q_{n} & \cdots & q_{2} &q_{1}\end{bmatrix}^{*}
\end{align}
%
with
%
\begin{align}
    q_{1} &= 0 \nonumber\\
    q_{k} &= \hat{f}_{k-1}' - \alpha_{k}y_{k}^{-1}\hat{f}_{k} \nonumber\\
    &=\brackets{1-\alpha_{k}y_{k}^{-1}}a_{k-1}
        + \eta_{k-1} - \alpha_{k}y_{k}^{-1}\eps_{k}
        \quad k\in\braces{2,\cdots,n} \nonumber\\
    q_{n+1} &= \hat{f}_{n}'.
\end{align}
%
The reason for this unusual ordering is because it makes the notation easier.
Then, the second term in Eq.~\eqref{eq:cvip_birkhoff_full_msn_total_terms}
becomes
%
\begin{align}
    \tilde{a}_{2} &= D_{s}^{-1}G^{*} \begin{bmatrix} 0 \\ q \end{bmatrix}
        \nonumber\\
    &= \begin{bmatrix} -s_{1}1^{-s}q_{1} \\ \vdots \\ -s_{n}n^{-s}q_{n} \\
        (n+1)^{-s}q_{n+1} \\ c_{n}(n+2)^{-s}q_{n} \\ \vdots \\
        c_{1}(2n+1)^{-s}q_{1} \end{bmatrix}.
    \label{eq:cvip_birkhoff_1D_full_q_coefs}
\end{align}
%
Now, we see
%
\begin{align}
    \abs{-s_{k}k^{-s}} &\le n^{-s} \nonumber\\
    \abs{c_{k}\parens{2n+2-k}^{-s}} & \le n^{-s}
\end{align}
%
and
%
\begin{align}
    \abs{1 - \alpha_{k}y_{k}^{-1}} &= \abs{2 - k + 2ns_{k}^{2}} \nonumber\\
        &\le k + 2ns_{k}^{2} \nonumber\\
    \abs{\alpha_{k}y_{k}^{-1}} &\le k + 2ns_{k}^{2},
\end{align}
%
implying
%
\begin{equation}
    \abs{-s_{k}k^{-s}q_{k}} \le \brackets{n^{-s}k + 2n^{1-s}s_{k}^{2}}
        \abs{a_{k-1}} + n^{-s}\abs{\eta_{k-1}}
        + \brackets{n^{-s}k +2n^{1-s}s_{k}^{2}}\abs{\eps_{k}}.
\end{equation}
%
%This exact bound holds for $\abs{c_{k}\parens{2n+2-k}q_{k}}$ except
%when $k=n+1$.
We also have the bound
%
\begin{align}
    \abs{c_{k}\parens{2n+2-k^{-s}q_{k}}}
    &\le \brackets{n^{-s}k + 2n^{1-s}s_{k}^{2}}
        \abs{a_{k-1}} + n^{-s}\abs{\eta_{k-1}}
        + \brackets{n^{-s}k +2n^{1-s}s_{k}^{2}}\abs{\eps_{k}}, \nonumber\\
    &\quad k\in\braces{1,\cdots,n}.
\end{align}
%
When $k=n+1$, we have the easier bound
%
\begin{equation}
    \abs{\parens{n+1}^{-s}q_{n+1}}
        \le n^{-s}\brackets{\abs{a_{n}} + \abs{\eta_{n}}}.
\end{equation}
%
All of the above work shows can be combined to show
%
\begin{align}
    \sum_{k=1}^{n} \abs{-s_{k}k^{-s}q_{k}} &\le
        \frac{\sqrt{\zeta(2s-2)}\norm{a}_{s}}{n^{s}}
        + 2\sqrt{\frac{2}{2s+1}}\frac{\norm{a}_{s}}{n^{2s-\frac{3}{2}}}
        + \frac{1}{\sqrt{2s-1}}\frac{\norm{a}_{s}}{n^{2s-\frac{1}{2}}}
            \nonumber\\
    &\quad+ \frac{1}{\sqrt{2s-1}}\frac{\norm{a}_{s}}{n^{2s-\frac{3}{2}}}
        + \frac{2}{\sqrt{2s-1}}\frac{\norm{a}_{s}}{n^{2s-\frac{3}{2}}}.
\end{align}
%
The same bound holds for $\sum_{k=1}^{n+1}\abs{c_{k}\parens{2n+2-k}^{-s}q_{k}}$.
All of these terms are $O(n^{-s+\frac{1}{2}})$
and, naturally, we have the restriction $s>\frac{3}{2}$.

Taken altogether, we see
%
\begin{align}
    \norm{a_{c} - \tilde{a}}_{p,1}
        &\le \norm{a_{c} - \tilde{a}_{1}}_{p,1} + \norm{\tilde{a}_{2}}_{p,1}
        \nonumber\\
    &\le \frac{C_{s}\norm{a}_{s}}{n^{s-\frac{1}{2}}},
\end{align}
%
with convergence when $\frac{3}{2}<s\le\sigma$.

From the above work, we can also compute the condition number.
First, we see
%
\begin{align}
    \norm{\begin{bmatrix}Y&\\A&B\end{bmatrix}
            \begin{bmatrix}a\\b\end{bmatrix}}_{2}^{2}
    &= \norm{Ya}_{2}^{2} + \norm{Aa + Yb}_{2}^{2} \nonumber\\
    &\le 25.
\end{align}
%
We also note we have $\norm{L}_{2}^{2}\ge\frac{1}{2}$.
We now look at the greater challenge of bounding the inverse,
and we use similar methods to those above:
%
\begin{align}
    \norm{\begin{bmatrix}Y&\\A&B\end{bmatrix}^{-1}
            \begin{bmatrix}a\\b\end{bmatrix}}_{2}^{2}
    &= ||Y^{-1}a||_{2}^{2} + ||B^{-1}(b - AY^{-1}a)||_{2}^{2} \nonumber\\
    &\le 2n^{2s} + \frac{2^{7}}{2s+1}n^{2s+1} \nonumber\\
    &\le 2^{8}n^{2s+1}.
\end{align}
%
Due to $Y$, we also have $||L^{-1}||_{2}^{2}\ge n^{2s}/2$.
Therefore, we have
%
\begin{equation}
    \frac{1}{2}n^{s} \le \kappa(L) \le 2^{4}n^{s+\frac{1}{2}}.
\end{equation}
%
Therefore, we have shown $\kappa(L) = \Omega(n^{s})$ and
$\kappa(L) = O(n^{s+\frac{1}{2}})$.




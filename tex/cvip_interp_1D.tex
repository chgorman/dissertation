\section{Proof of 1D Interpolation for polynomials of degree $2n$}
\label{sec:cvip_interp_1D}

We begin by computing the error. First, we have the MSN coefficients
%
\begin{align}
    \tilde{a} &= D_{s}^{-1}G^{*}\begin{bmatrix} Y^{-1}\hat{f} \\ 0 \end{bmatrix}
        \nonumber\\
    &= \begin{bmatrix} c_{1}^{2}\hat{f}_{1} \\ \vdots \\ c_{n}^{2}\hat{f}_{n} \\
        0 \\  s_{n}^{2}\hat{f}_{n} \\ \vdots \\ s_{1}^{2}\hat{f}_{1}
        \end{bmatrix}.
    \label{eq:cvip_interp_1D_msn_coefs}
\end{align}
%
From Eqs.~\eqref{eq:v1_I_2n_Givens_cs_def} and \eqref{eq:v1_I_2n_Givens_y_def}
in Sec.~\ref{sec:CV_1D_2n}, we recall
%
\begin{align}
    \tau_{k} &= \parens{\frac{k}{2n+2-k}}^{s} \nonumber\\
    c_{k} &= \frac{1}{\sqrt{1+\tau^{2}}} \nonumber\\
    s_{k} &= \tau_{k}c_{k} \nonumber\\
    y_{k} &= \frac{1}{c_{k}k^{s}}.
\end{align}
%
It is clear
%
\begin{align}
    c_{k} & \le 1 \nonumber\\
    \tau_{k} &\le \frac{k^{s}}{n^{s}} \nonumber\\
    s_{k} &\le \tau_{k} \nonumber\\
    y_{k} &\ge k^{-s} \nonumber\\
    y_{k} &\le \sqrt{2}k^{-s}.
\end{align}
%
As stated in Sec.~\ref{sec:cvip_main}, we only need to focus on computing
%
\begin{align}
    \norm{a_{c} - \tilde{a}}_{p,1}
        &\le \sum_{k=1}^{n}|a_{k-1}-c_{k}^{2}\hat{f}_{k}|
            + \abs{a_{n}} + \sum_{k=1}^{n}|a_{2n+1-k}-s_{k}^{2}\hat{f}_{k}|
            \nonumber\\
        &\le 2\sum_{k=1}^{n}s_{k}^{2}\abs{a_{k-1}}
            + 2\sum_{k=n}^{\infty}\abs{a_{k}}.
\end{align}

We look at these terms separately. First, we see
%
\begin{align}
    \sum_{k=1}^{n}s_{k}^{2}\abs{a_{k-1}} &\le
        \frac{1}{n^{2s}} \sum_{k=1}^{n}k^{s}\cdot k^{s}\abs{a_{k-1}} \nonumber\\
    &\le \frac{1}{n^{2s}}\sqrt{\sum_{k=1}^{n} k^{2s}}\norm{a}_{s}
        \nonumber\\
    &\le \sqrt{\frac{2}{2s+1}}\frac{\norm{a}_{s}}{n^{s-\frac{1}{2}}}.
    \label{eq:cvip_interp_1D_sine_bound}
\end{align}
%
Here, the bound in the last equality follows from Eq.~\eqref{eq:bound_xp_01}.
Similarly,
%
\begin{equation}
    \sum_{k=n}^{\infty}\abs{a_{k}}
        \le \frac{1}{\sqrt{2s-1}}\frac{\norm{a}_{s}}{n^{s-\frac{1}{2}}},
    \label{eq:cvip_interp_1D_tail_bound}
\end{equation}
%
which follows from Eq.~\eqref{eq:cvip_main_large_bound}.
Finally, we have the uniform bound
%
\begin{equation}
    \norm{f-p_{n}}_{\infty,[-1,1]}
        \le \frac{C_{s}}{n^{s-\frac{1}{2}}}\norm{a}_{s}.
\end{equation}
%
Thus, $\norm{f-p_{n}}_{\infty}\to0$ as $n\to\infty$
when $\frac{1}{2} < s \le \sigma$.



From the above work, we can also compute the condition number of
the matrix $WD_{s}^{-1}$. It is clear
%
\begin{equation}
    \frac{n^{s}}{\sqrt{2}}\le y_{1}y_{n}^{-1} \le \sqrt{2}n^{s},
\end{equation}
%
so it follows $\kappa_{2}(L) = \Theta(n^{s})$.



The above work suggests that we must choose $s\le\sigma$, but
it turns out that this restriction is not necessary.
To see this, we will look at Eqs.~\eqref{eq:cvip_interp_1D_sine_bound}
and \eqref{eq:cvip_interp_1D_tail_bound} to see where we
were too loose in our bounds.
We will use the bound
%
\begin{equation}
    \parens{1+k}^{\sigma}\abs{a_{k}} \le \norm{a}_{\sigma}.
\end{equation}
%
In Eq.~\eqref{eq:cvip_interp_1D_sine_bound}, we have
%
\begin{align}
    \sum_{k=1}^{n}s_{k}^{2}\abs{a_{k-1}} &\le
        \frac{1}{n^{2s}} \sum_{k=1}^{n}k^{2s}\abs{a_{k-1}} \nonumber\\
    &\le \frac{\norm{a}_{\sigma}}{n^{2s}}\sum_{k=1}^{n} k^{2s-\sigma}
        \nonumber\\
    &\le \frac{2\norm{a}_{\sigma}}{2s-\sigma+1}\frac{1}{n^{\sigma-1}}.
    %\label{eq:cvip_interp_1D_sine_bound}
\end{align}
%
Here, we are assuming $2s-\sigma>-1$; otherwise, we have the bound
\begin{equation}
    \sum_{k=1}^{n}s_{k}^{2}\abs{a_{k-1}} \le C_{s,\sigma}\frac{\log n}{n^{2s}}.
\end{equation}
%
Thus, the requirement here is that $\sigma>1$.
Similarly, Eq.~\eqref{eq:cvip_interp_1D_tail_bound} can be replaced with
%
\begin{equation}
    \sum_{k=n}^{\infty}\abs{a_{k}}
        \le \frac{1}{\sqrt{2\sigma-1}}\frac{\norm{a}_{\sigma}}{
        n^{\sigma-\frac{1}{2}}},
\end{equation}
%
so that $\sigma>\frac{1}{2}$.

Taken together, we have improved our bound for convergence to
%
\begin{equation}
    \norm{f-p_{n}}_{\infty,[-1,1]}
        \le \frac{C_{s,\sigma}}{n^{\sigma-1}}\norm{a}_{\sigma}
\end{equation}
%
when $\sigma>1$, and we can choose $s>\frac{1}{2}$.
Whether this can be extended to the case when $\sigma\in(\frac{1}{2},1]$
will be looked at in the future.
This also shows that the results in future sections likely
could be strengthened.




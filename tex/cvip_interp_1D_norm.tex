\section{Proof of Norm Convergence for 1D Interpolation of polynomials
    of degree $2n$}
\label{sec:cvip_interp_1D_norm}

In this section we determine when our MSN approximation controlling the
$s$th derivative will converge to the true solution in $\norm{\cdot}_{\tau}$
when $\norm{a}_{\sigma}<\infty$.
We will be assuming throughout that $\sigma>\frac{1}{2}$
and $s>\frac{1}{2}$ as this simplifies some of the assumptions.
We will discuss the situation when $\sigma\in[0,\frac{1}{2}]$ at
the end of this section.

If $\tilde{a}$ are the coefficients of the $2n$ MSN solution controlling
the $s$ derivative, then we see
%
\begin{align}
    \norm{a-\tilde{a}}_{\tau}^{2} &= \norm{a-a_{c}}_{\tau}^{2}
        + \norm{\tilde{a}-a_{c}}_{\tau}^{2} \nonumber\\
    &= \sum_{k=2n+1}^{\infty} \parens{1+k}^{2\tau}a_{k}^{2}
        + \norm{\tilde{a}-a_{c}}_{\tau}^{2} \nonumber\\
    &\le \norm{\tilde{a}-a_{c}}_{\tau}^{2}
        + \frac{\norm{a}_{\sigma}^{2}}{2\sigma-2\tau-1}
        \frac{1}{(2n)^{2\sigma-2\tau-1}}.
\end{align}
%
As before, $a_{c}$ is the true solution chopped to degree $2n$.
Thus, we only need to look at bounding $\norm{\tilde{a}-a_{c}}_{\tau}$.
Furthermore, the above inequality shows we must have $\tau<\sigma-\frac{1}{2}$.

Before getting into the main details, we will need to use the bound
%
\begin{align}
    \sum_{k=1}^{n}\abs{\eps_{k}}^{2}
        &\le \parens{\max_{j} \abs{\eps_{j}}}
            \sum_{k=1}^{n}\abs{\eps_{k}} \nonumber\\
    &\le \brackets{\sum_{k=1}^{n}\abs{\eps_{k}}}^{2} \nonumber\\
    &\le \frac{\norm{a}_{\sigma}^{2}}{2\sigma-1}\frac{1}{n^{2\sigma-1}},
\end{align}
%
where $\eps_{k}$ is defined in Eq.~\eqref{eq:cvip_idct_fhat}
and we have the useful bound
%
\begin{equation}
    \sum_{k=1}^{n} \abs{\eps_{k}} \le
        \frac{1}{\sqrt{2\sigma-1}}\frac{\norm{a}_{\sigma}}{
        n^{\sigma-\frac{1}{2}}}
\end{equation}
%
from Eq.~\eqref{eq:cvip_eps_sum}.
This bound is just the fact
$\norm{x}_{2}^{2} \le \norm{x}_{1}\norm{x}_{\infty}$.
It turns out this bound is the only time that we must have
$\sigma>\frac{1}{2}$.

We know
%
\begin{align}
    \norm{\tilde{a}-a_{c}}_{\tau}^{2}
        &= \sum_{k=1}^{n}k^{2\tau}(a_{k-1} - c_{k}^{2}\hat{f}_{k})^{2}
            + \parens{1+n}^{2\tau}a_{n}^{2} \nonumber\\
            &\quad + \sum_{k=n+1}^{2n}\parens{1+k}^{2\tau}(a_{k}
                -s_{2n+1-k}^{2}\hat{f}_{2n+1-k})^{2}.
\end{align}
%
We will begin by looking at the first sum and then the remaining terms.

From our previous work, we know
%
\begin{equation}
    \parens{a_{k-1}-c_{k}^{2}\hat{f}_{k}}^{2}
        \le 2\parens{s_{k}^{4}a_{k-1}^{2} + \abs{\eps_{k}}^{2}}.
\end{equation}
%
The following summation bound is clear:
%
\begin{align}
    \sum_{k=1}^{n}k^{2\tau}s_{k}^{4}a_{k-1}^{2}
        &\le \frac{\norm{a}_{\sigma}^{2}}{n^{4s}}
        \sum_{k=1}^{n}k^{2\tau+4s-2\sigma} \nonumber\\
        &\le \frac{2\norm{a}_{\sigma}^{2}}{2\tau+4s-2\sigma+1}
        \frac{1}{n^{2\sigma-2\tau-1}}.
    \label{eq:cvip_norm_conv_s_constraint_example_bound}
\end{align}
%
The sum that we bound has the form as given in Eq.~\eqref{eq:bound_xp_01};
we are only reproducing the bound when $2\tau+4s-2\sigma>-1$ for
simplicity, as the other cases are similar and there is still decay in $n$:
$O(n^{-4s}\ln n)$ or $O(n^{-4s})$.
Additionally, We also have the bound
%
\begin{align}
    \sum_{k=1}^{n}k^{2\tau}\abs{\eps_{k}}^{2}
        &\le n^{2\tau}\sum_{k=1}^{n}\abs{\eps_{k}}^{2} \nonumber\\
    &\le \frac{\norm{a}_{\sigma}^{2}}{2\sigma-1}\frac{1}{
        n^{2\sigma-2\tau-1}}.
\end{align}

We now look at
%
\begin{align}
    &\parens{1+n}^{2\tau}a_{n}^{2}
        + \sum_{k=n+1}^{2n}\parens{1+k}^{2\tau}(a_{k}
                -s_{2n+1-k}^{2}\hat{f}_{2n+1-k})^{2} \nonumber\\
    &\quad\le 2\sum_{k=n}^{2n}\parens{1+k}^{2\tau}a_{k}^{2}
        + 2\sum_{k=1}^{n}\parens{2n+2-k}^{2\tau}s_{k}^{4}\hat{f}_{k}^{2}.
\end{align}
%
It is clear that the first sum reduces to
%
\begin{align}
    \sum_{k=n}^{2n}\parens{1+k}^{2\tau}a_{k}^{2}
        &\le \norm{a}_{\sigma}^{2}\sum_{k=1}^{n+1}[n+k]^{2\tau-2\sigma}
        \nonumber\\
    &\le \frac{2\norm{a}_{\sigma}^{2}}{2\sigma-2\tau-1}\frac{1}{
        n^{2\sigma-2\tau-1}},
\end{align}
%
where we are using the bound form Eq.~\eqref{eq:ubound_xp_12_pneg}.
In the second sum, we see
%
\begin{align}
    \sum_{k=1}^{n}\parens{2n+2-k}^{2\tau}s_{k}^{4}\hat{f}_{k}^{2}
        &\le \parens{2n+1}^{2\tau}\sum_{k=1}^{n}s_{k}^{4}\hat{f}_{k}^{2}
        \nonumber\\
    &\le 2\parens{2n+1}^{2\tau}\sum_{k=1}^{n}s_{k}^{4}\parens{
        a_{k}^{2}+\abs{\eps_{k}}^{2}}.
\end{align}
%
The first portion of the sum simplifies as follows:
%
\begin{align}
    \sum_{k=1}^{n}s_{k}^{4}a_{k}^{2} &\le \frac{\norm{a}_{\sigma}^{2}}{n^{4s}}
        \sum_{k=1}^{n}k^{4s-2\sigma} \nonumber\\
    &\le \frac{\norm{a}_{\sigma}^{2}}{4s-2\sigma+1}\frac{1}{n^{2\sigma-1}}.
\end{align}
%
Similarly, we have
%
\begin{align}
    \sum_{k=1}^{n}s_{k}^{4}\abs{\eps_{k}}^{2}
        &\le \sum_{k=1}^{n}\abs{\eps_{k}}^{2} \nonumber\\
    &\le \frac{\norm{a}_{\sigma}^{2}}{2\sigma-1}\frac{1}{n^{2\sigma-1}}.
\end{align}
%
If we combine these inequalities, we find
%
\begin{equation}
    \sum_{k=1}^{n}\parens{2n+2-k}^{2\tau}s_{k}^{4}\hat{f}_{k}^{2}
        \le \frac{C_{\tau,\sigma,s}\norm{a}_{\sigma}^{2}}{n^{2\sigma-2\tau-1}}.
\end{equation}

Taken together, we have
%
\begin{align}
    \norm{a-\tilde{a}}_{\tau}^{2} &= \norm{a-a_{c}}_{\tau}^{2}
        + \norm{\tilde{a}-a_{c}}_{\tau}^{2} \nonumber\\
    &\le \frac{C_{s,\sigma,\tau}\norm{a}_{\sigma}^{2}}{n^{2\sigma-2\tau-1}},
\end{align}
%
with the restrictions $\sigma>\frac{1}{2}$ and $\tau<\sigma-\frac{1}{2}$.

As we can see, the only instance where we require $\sigma>\frac{1}{2}$
occurs in bounding $\sum_{k=1}^{n}\abs{\eps_{k}}^{2}$.
When proving convergence in the original MSN interpolation
papers~\cite{msnInterp,msnBirkhoff},
dyadic sums were used in order to determine a sufficient interpolation degree
to ensure bounded derivative norm.
The interpolation examples in Chapter~\ref{chap:func_interp}
show that even when the function is discontinuous, we obtain
convergence to the underlying solution where the function in continuous.
This suggests that the convergence in norm results may be extended
to the case when $\sigma\in\brackets{0,\frac{1}{2}}$ (e.g.~when
the function is integrable but not continuous) but the proof of this
may require careful analysis due to the possibly non-absolutely
converging terms $\eps_{k}$; this will be looked at in future work.
It is well-known conditionally convergent series are challenging
to work with~\cite{baby_rudin,knopp_series}.

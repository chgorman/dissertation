\section{Main Idea}
\label{sec:cvip_main}

In this chapter we will compute error bounds for the MSN method.
In \cite{msnInterp}, Chandrasekaran et~al.~showed the MSN
solution was guaranteed to converge to the underlying solution
based on a compactness argument by showing the polynomial
approximations had bounded derivative in combination with the
interpolation conditions and the Arzel\`{a}-Ascoli theorem.
We take a different route by carefully looking at the
linear system and solving for the interpolation coefficients,
which is possible due to its structured nature.

Given
%
\begin{equation}
    f(x) = \sum_{k=0}^{\infty} a_{k}T_{k}(x),
\end{equation}
%
we know
%
\begin{align}
    \norm{f}_{\infty,[-1,1]} &\le \sum_{k=0}^{\infty} \abs{a_{k}} \nonumber\\
        &\equiv \norm{a}_{p,1},
\end{align}
%
where we obtain equality of norm at $x=0$ and $a_{k}$ all the same sign.
Naturally, $\norm{a}_{p,1}$ is the 1-norm of the sequence of coefficients.
We find
%
\begin{align}
    \sum_{k=N}^{\infty} \abs{a_{k}}
        &\le \sqrt{\sum_{k=N}^{\infty} \parens{k+1}^{-2s}}\norm{a}_{s}
        \nonumber\\
    &\le \frac{1}{\sqrt{2s-1}}\frac{\norm{a}_{s}}{N^{s-\frac{1}{2}}}.
    \label{eq:cvip_main_large_bound}
\end{align}
%
If $a_{c}$ are the true coefficients chopped to $N$ and $\tilde{a}$
are the coefficients of a polynomial approximation $p_{N}$ of degree $N$,
then we see
%
\begin{align}
    \norm{f-p_{N}}_{\infty,[-1,1]} &\le \norm{a-\tilde{a}}_{p,1} \nonumber\\
        &= \norm{a_{c}-\tilde{a}}_{p,1} + \norm{a-a_{c}}_{p,1} \nonumber\\
    &\le \norm{a_{c}-\tilde{a}}_{p,1}
        + \frac{C_{s}\norm{a}_{s}}{N^{s-\frac{1}{2}}}.
\end{align}
%
Because of this, for our MSN approximation $p_{n}$,
the best asymptotic error we can have is
%
\begin{equation}
    \norm{f-p_{n}}_{\infty,[-1,1]}
        \le \frac{C_{s}\norm{a}_{s}}{n^{s-\frac{1}{2}}}
\end{equation}
%
using this method.
Therefore, we will focus on bounding $\norm{a_{c}-\tilde{a}}_{p,1}$.


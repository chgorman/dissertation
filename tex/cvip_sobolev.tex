\section{Sobolev Embedding Theorems and Related Work}
\label{sec:cvip_sobolev}

This work has focused on functions with bounded
Sobolev norm, but because it is usually more convenient
to think it terms of continuous
and continuously-differentiable functions, we prove some embedding theorems.

We assume
%
\begin{equation}
    g(\theta) = \sum_{k\in\Z} a_{k}e^{ik\theta}.
\end{equation}
%
In this case,
%
\begin{align}
    \sum_{k=-N}^{N} \abs{a_{k}}
        &= \sum_{k=-N}^{N} \parens{1+\abs{k}}^{-s}
            \brackets{\parens{1+\abs{k}}^{s}\abs{a_{k}}} \nonumber\\
    &\le \sqrt{\sum_{k=-N}^{N}\parens{1+\abs{k}}^{-2s}} \norm{a}_{s} \nonumber\\
    &\le \sqrt{2\zeta(2s)}\norm{a}_{s}.
\end{align}
%
Thus, if $s>\frac{1}{2}$ and $\norm{a}_{s}<\infty$, this upper bound
is finite and holds for all $N$, so $g(\theta)$ is continuous.
Similarly, for an integer $m\ge1$, we have
%
\begin{align}
    \sum_{\substack{k=-N\\k\ne0}}^{N} \abs{k}^{m}\abs{a_{k}}
        &\le \sum_{k=-N}^{N}\parens{1+\abs{k}}^{m}\abs{a_{k}} \nonumber\\
    &\le \sqrt{2\zeta(2s-2m)}\norm{a}_{s}.
\end{align}
%
Thus, for $s>m+\frac{1}{2}$ and $\norm{a}_{s}<\infty$, the above bound
is finite and holds for all $N$, showing $g^{(m)}$ is continuous.
Therefore, we have shown
%
\begin{equation}
    H_{s} \subseteq C^{m},\quad s>m+\frac{1}{2}
\end{equation}
%
for integers $m\ge0$.

In the other direction, if $g\in C^{m,\alpha}$, so that $g^{(m)}$
is $\alpha$-H\"{o}lder continuous with constant $L$, then
with integration by parts we find
%
\begin{align}
    a_{k} &= \frac{1}{2\pi}\int_{-\pi}^{\pi} g(\theta)e^{-ik\theta}d\theta
        \nonumber\\
    &= \frac{1}{2\pi\parens{ik}^{m}}
        \int_{-\pi}^{\pi} g^{(m)}(\theta)e^{-ik\theta}d\theta.
\end{align}
%
Similarly, by changing the limits of integration, we see
%
\begin{equation}
    a_{k} = -\frac{1}{2\pi\parens{ik}^{m}} \int_{-\pi}^{\pi}
        g^{(m)}\parens{\theta+\frac{\pi}{k}}e^{-ik\theta}d\theta,
    \quad k\ne0.
\end{equation}
%
Therefore, we find
%
\begin{align}
    \abs{a_{k}} &\le \frac{1}{4\pi\abs{k}^{m}}\int_{-\pi}^{\pi}
        \abs{g(\theta)-g\parens{\theta+\frac{\pi}{k}}}d\theta \nonumber\\
    &\le \frac{L\pi^{\alpha}}{2\abs{k}^{m+\alpha}}.
\end{align}
%
Thus, we see $a_{k} = O(k^{-m-\alpha})$.
We let $C = \max\braces{\abs{a_{0}},L\pi^{\alpha}/2}$.
In this case
%
\begin{align}
    \sum_{k=-N}^{N} \parens{1+\abs{k}}^{2s} \abs{a_{k}}^{2}
        &\le C^{2} + C^{2}\sum_{\substack{k=-N\\n\ne0}}^{N}
            \parens{1+\abs{k}}^{2s} \frac{1}{\abs{k}^{2m+2\alpha}} \nonumber\\
    &\le C^{2}\brackets{1 + 2^{2s+1}\sum_{k=1}^{N} \abs{k}^{2s-2m-2\alpha}}
        \nonumber\\
    &\le C^{2}\brackets{1 + 2^{2s+1}\zeta(2m+2\alpha-2s)}.
\end{align}
%
The above bound is holds for all $N$ and is finite so long as
$s<m+\alpha+\frac{1}{2}$, which implies $\norm{a}_{s} < \infty$.
Thus, we have shown
%
\begin{equation}
    C^{m,\alpha}\subseteq H_{s}, \quad s<m+\alpha+\frac{1}{2}.
\end{equation}
%
This last bound can be slightly improved by looking
at functions of bounded variation (see \cite{ATAP,zygmund}),
but we will not pursue the matter here.




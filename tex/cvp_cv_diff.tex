\section{Differentiation of Chebyshev Polynomials}
\label{sec:CV_diff}

We would eventually like to solve ordinary and partial differential equations
(ODEs and PDEs), so we must work with derivatives of Chebyshev polynomials.
We define the infinite Chebyshev-Vandermonde derivative matrix
$V_{\infty}'$ by
%
\begin{equation}
\brackets{V_{\infty}'}_{ij}
    = T_{j-1}'\parens{z_{i}^{n}},\quad i\in\braces{1,\cdots,n}
\quad j\in\N.
\end{equation}
%
Any other Chebyshev-Vandermonde matrix $V'$ will contain
a finite number of columns of $V_{\infty}'$. 

We prove the following theorem:

\begin{thm}[Derivative of Chebyshev Polynomials]
We have the following relationship between Chebyshev polynomials
and their derivatives:
%
\begin{align}
    T_{2n+1}' &= 2\parens{2n+1}\parens{T_{2n} + T_{2n-2} + \cdots + T_{2} +
                    \frac{1}{2}T_{0}} \nonumber\\
    T_{2n}' &= 2\parens{2n}\parens{T_{2n-1} + T_{2n-3} + \cdots + T_{1}}.
\end{align}
\end{thm}

\begin{proof}
We proceed by induction. First, from the recurrence relation
in Eq.~\eqref{eq:cheby_rec_rel}, we see
%
\begin{align}
    T_{0}(x) &= 1 \nonumber\\
    T_{1}(x) &= x \nonumber\\
    T_{2}(x) &= 2x^{2} - 1 \nonumber\\
    T_{3}(x) &= 4x^{3} - 3x.
\end{align}
%
Direct computation shows us
%
\begin{align}
    T_{0}'(x) &= 0 \nonumber\\
        &= 2(2\cdot0) \nonumber\\
    T_{1}'(x) &= 1 \nonumber\\
        &= 2(2\cdot0+1)\frac{1}{2}T_{0}(x) \nonumber\\
    T_{2}'(x) &= 4x \nonumber\\
        &= 2(2\cdot1)T_{1}(x) \nonumber\\
    T_{3}'(x) &= 12x^{2} - 3 \nonumber\\
        &= 2(2\cdot1+1)\brackets{T_{2}(x) + \frac{1}{2}T_{0}(x)}.
\end{align}
%
Thus, we have proven the base cases $n=0$ and $n=1$.

Assume $n\ge2$ and that the relationship holds for
for $T_{2k}'$ and $T_{2k+1}'$ for all $k<n$.
From the recurrence relationship of Chebyshev polynomials
and recognizing $T_{1}(x) = x$,
we see
%
\begin{align}
    T_{2n}' &= 2T_{2n-1} + 2T_{1}T_{2n-1}' - T_{2n-2}' \nonumber\\
    &= 2T_{2n-1} + 2T_{1}\braces{
        2\parens{2n-1}\brackets{T_{2n-2} + T_{2n-4} + \cdots + T_{2} +
        \frac{1}{2}T_{0}}}
        \nonumber\\
        &\quad - 2\parens{2n-2}\brackets{T_{2n-3} + T_{2n-5} + \cdots + T_{1}}
        \nonumber\\
    &= 2T_{2n-1} + 2\parens{2n-1}\brackets{T_{2n-1} + 2T_{2n-3} + \cdots
        + 2T_{3} + 2T_{1}} \nonumber\\
        &\quad - 2\parens{2n-2}\brackets{T_{2n-3} + T_{2n-5} + \cdots + T_{1}}
        \nonumber\\
    &= 2\parens{2n}\brackets{T_{2n-1} + T_{2n-3} + \cdots + T_{1}}.
\end{align}
%
Similarly, we have
%
\begin{align}
    T_{2n+1}' &= 2T_{2n} + 2T_{1}T_{2n}' - T_{2n-1}' \nonumber\\
        &= 2T_{2n} + 2T_{1}\braces{2(2n)\brackets{T_{2n-1} + T_{2n-3}
            + \cdots + T_{1}}} \nonumber\\
    &\quad - 2\parens{2n-1}\brackets{T_{2n-2} + T_{2n-4} + \cdots + T_{2}
        + \frac{1}{2}T_{0}} \nonumber\\
    &= 2T_{2n} + (2n)\brackets{T_{2n} + 2T_{2n-2} + 2T_{2n-4} + \cdots
        + 2T_{2} + T_{0}} \nonumber\\
    &\quad - 2\parens{2n-1}\brackets{T_{2n-2} + T_{2n-4} + \cdots + T_{2}
        + \frac{1}{2}T_{0}} \nonumber\\
    &= 2\parens{2n+1}\brackets{T_{2n} + T_{2n-2} + \cdots + T_{2} +
        \frac{1}{2}T_{0}}.
\end{align}
%
This is the desired result.
\end{proof}



This allows us to write
%
\begin{equation}
    V_{\infty}' = V_{\infty}D_{\infty},
\end{equation}

\noindent
where
%
\begin{equation}
    D_{\infty} = 2\begin{bmatrix}
            0 & \frac{1}{2} & 0 & \frac{1}{2} & 0 & \frac{1}{2} & 0 & \cdots \\
            0 & 0 & 1 & 0           & 1 & 0           & 1 & \cdots \\
            0 & 0 & 0 & 1           & 0 & 1           & 0 & \cdots \\
            \vdots & \ddots & \ddots & \ddots & \ddots & \ddots & \ddots
                   & \ddots \\
        \end{bmatrix}
        \diag\parens{0,1,2,3,4,\cdots}.
\end{equation}
%
Explicitly, we see
%
\begin{align}
    \brackets{D_{\infty}}_{1,2k} &= 2k-1,\quad k\in\N \nonumber\\
    \brackets{D_{\infty}}_{j,j-1+2k} &= 2\parens{2k+j-2},
        \quad j\in\braces{2,3,\cdots},\quad k\in\N.
\end{align}
%
All other entries are 0.

When we have $n=9$ rows and $2n+1$ columns, we see
%
\begin{equation}
    WD = \begin{bmatrix}
    0 & 1 & & 3 & & 5 & & 7 & & 9 & & 11 & & 13 & & 15 & & 17 & 0 \\
    & & 4 & & 8 & & 12 & & 16 & & 20 & & 24 & & 28 & & 32 & & \\
    & & & 6 & & 10 & & 14 & & 18 & & 22 & & 26 & & 30 & & & \\
    & & & & 8 & & 12 & & 16 & & 20 & & 24 & & 28 & & & & \\
    & & & & & 10 & & 14 & & 18 & & 22 & & 26 & & & & & \\
    & & & & & & 12 & & 16 & & 20 & & 24 & & & & & & \\
    & & & & & & & 14 & & 18 & & 22 & & & & & & & \\
    & & & & & & & & 16 & & 20 & & & & & & & & \\
    & & & & & & & & & 18 & & & & & & & & & \\
    \end{bmatrix}.
\end{equation}
%
This pattern holds in general.
For a simpler structure, we let
%
\begin{equation}
    U = \begin{bmatrix}
    1  &    & -1 \\
       &  1 &    & -1 \\
       &    &  1 &    & -1 \\
       &    &    &  1 &    & -1 \\
       &    &    &    &  \ddots &    & \ddots \\
       &    &    &    &    &  1 &    & -1 \\
       &    &    &    &    &    &  1 &    & -1 \\
       &    &    &    &    &    &    &  1 &    \\
       &    &    &    &    &    &    &    &  1 \\
    \end{bmatrix}
    \diag\parens{1,\frac{1}{2},\cdots,\frac{1}{2}}.
    \label{eq:vand_U_mat}
\end{equation}
%
Then, for general $n$ rows and $2n+1$ columns, we have
%
\begin{equation}
    UWD = \begin{bmatrix}
    0 & 1 &   &   &   &   &   &   &   &    &    &2n-1&  0 \\
      &   & 2 &   &   &   &   &   &   &    &2n-2&    &    \\
      &  & &\ddots&   &   &   &   &   &\iddots& &    &    \\
      &   &   &   &n-2&   &   &   &n+2&    &    &    &    \\
      &   &   &   &   &n-1&   &n+1&   &    &    &    &    \\
      &   &   &   &   &   & n &   &   &    &    &    &    \\
    \end{bmatrix}.
    \label{eq:CV_UWD_mat}
\end{equation}

\section{\CV{} Matrix Properties}
\label{sec:CV_mat}

Recalling our definitions of the Chebyshev polynomials $T_{k}$
from Sec.~\ref{ssec:kar_cheby},
we define the infinite \CV{} matrix $V_{\infty}$ by
%
\begin{equation}
    \brackets{V_{\infty}}_{ij}
        = T_{j-1}\parens{z_{i}^{n}},\quad i\in\braces{1,\cdots,n}
    \quad j\in\N.
\end{equation}
%
Any other \CV{} matrix $V$ will contain
a finite number of columns of $V_{\infty}$.
Furthermore, we let
%
\begin{equation}
\label{eq:C_def}
    C_{ij} = T_{j-1}\parens{z_{i}^{n}},\quad i,j\in\braces{1,\cdots,n};
\end{equation}
%
that is, $C$ is just the first $n\times n$ block of $V_{\infty}$.
This $C$ corresponds to our choice of DCT, discussed in Sec.~\ref{ssec:kar_dct}.
We show
%
\begin{align}
    C^{*}V_{\infty} &= F\begin{bmatrix}
    I & 0 & -\Lambda & 0 & \Lambda & 0 & -\Lambda & 0 & \cdots \\
    \end{bmatrix} \nonumber\\
     &= FW_{\infty},
\end{align}
%
where $I$ is the $n\times n$ identity matrix, $0$ is a column of zeros,
%
\begin{equation}
    F = \frac{n}{2}\diag\parens{2, 1, 1, \cdots, 1},
\end{equation}
%
and
%
\begin{equation}
    \Lambda = \begin{bmatrix}
    & & & & 1 & & & & \\
    & & & 1 & & 1 & & & \\
    & & \iddots & & & & \ddots & & \\
    & 1 & & & & & & 1 & \\
    1 & & & & & & & & 1 \\
    \end{bmatrix},
\end{equation}
%
an $n\times 2n-1$ matrix. We see that the zero columns of $W_{\infty}$
are located at $n$, $3n$, $5n$, \dots, and that the
first row of $W$ has $\pm1$ located at column $0$, $2n$, $4n$, \dots.
Explicitly, we have
%
\begin{align}
    \brackets{W_{\infty}}_{k,k+4(\ell-1)n} &= 1,
        \quad k\in\braces{1,\cdots,n}, \quad \ell\in\N \nonumber\\
    \brackets{W_{\infty}}_{k,(4\ell-2)n+2-k} &= -1,
        \quad k\in\braces{2,\cdots,n}, \quad \ell\in\N \nonumber\\
    \brackets{W_{\infty}}_{k,k+(4\ell-2)n} &= -1,\quad k\in\braces{1,\cdots,n},
        \quad \ell\in\N \nonumber\\
    \brackets{W_{\infty}}_{k,4\ell n+2-k} &= 1,
        \quad k\in\braces{2,\cdots,n}, \quad \ell\in\N.
\end{align}

We prove this.
First, we have
%
\begin{align}
    \brackets{C^{*}V_{\infty}}_{i+1,j+1}
        &= \sum_{k=1}^{n} T_{i}\parens{z_{k}^{n}}
        T_{j}\parens{z_{k}^{n}} \nonumber\\
     &= \sum_{k=1}^{n}
        \cos\brackets{\frac{i\pi}{n}\parens{n - k + \frac{1}{2}}}
        \cos\brackets{\frac{j\pi}{n}\parens{n - k + \frac{1}{2}}}
        \nonumber \\
     &= \frac{1}{2}\sum_{k=1}^{n}\braces{
            \cos\brackets{\frac{i + j}{n}\pi\parens{n - k + \frac{1}{2}}}
            + \cos\brackets{\frac{i - j}{n}\pi\parens{n - k + \frac{1}{2}}}
        } \nonumber \\
     &= \frac{1}{2}\sum_{k=0}^{n-1}\braces{
            \cos\brackets{\frac{i + j}{2n}\pi\parens{2k + 1}}
            + \cos\brackets{\frac{i - j}{2n}\pi\parens{2k + 1}}
        },
\end{align}
%
where the reductions are standard product-to-sum trigonometric identities
and reversing the sum. As we can see, we are finished if we can sum
%
\begin{equation}
    \sum_{k=0}^{n-1} \cos\brackets{\frac{\alpha\pi}{2n}\parens{2k + 1}}.
\end{equation}

If $\alpha\in\braces{0,\pm4\pi,\pm8\pi,\cdots}$ the above
sum is $n$, while if $\alpha\in\braces{\pm2\pi,\pm6\pi,\pm10\pi,\cdots}$,
the sum is $-n$. We show if $\alpha$ is any other integer
then the previous sum is $0$.
To see this, we complexify the situation.
Letting
%
\begin{align}
\xi &= \exp\brackets{i\frac{\alpha\pi}{n}} \nonumber\\
\eta &= \exp\brackets{i\frac{\alpha\pi}{2n}},
\end{align}
%
so that $\xi$ is the primitive $n$th root of unity and $\eta$ is
the primitive $2n$th root of unity, we see
%
\begin{equation}
\cos\brackets{\frac{\alpha\pi}{2n}\parens{2k + 1}} = \Re\parens{\xi^{k}\eta}.
\end{equation}

Therefore, it follows
%
\begin{align}
    \sum_{k=0}^{n-1}\xi^{k}\eta &= \eta\frac{1 - \xi^{n}}{1 - \xi} \nonumber\\
     &= \frac{1 - \parens{-1}^{\alpha}}{\eta^{-1} - \xi\eta^{-1}} \nonumber\\
     &= \frac{1}{2i}\frac{\parens{-1}^{\alpha} - 1}{
                \sin\parens{\frac{\alpha\pi}{2n}}}.
\end{align}
%
This is purely imaginary when $\alpha\ne 2n\gamma$
for $\gamma\in\Z$, so we have proven
%
\begin{equation}
\sum_{k=0}^{n-1} \cos\brackets{\frac{\alpha\pi}{2n}\parens{2k + 1}}
 = \begin{cases}
    n &\quad \alpha = 4n\gamma, \quad \gamma\in\Z \\
    -n &\quad \alpha = 4n\gamma + 2, \quad \gamma\in\Z \\
    0 &\quad\text{otherwise}
\end{cases}.
\end{equation}
%
From here, we see when $i,j\in\braces{0,1,\cdots,n-1}$, we have
%
\begin{equation}
\brackets{C^{*}V}_{ij} = \begin{cases}
    n &\quad i = j = 0 \\
    \frac{n}{2} &\quad i = j \ne 0 \\
    0 &\quad \text{otherwise}
\end{cases}.
\end{equation}
%
Furthermore, we see with $i,j\in\braces{1,2,\cdots,n-1}$, we have
%
\begin{equation}
\brackets{C^{*}V_{\infty}}_{i,n+j} = \begin{cases}
    -\frac{n}{2} &\quad i = n - j \\
    0 &\quad \text{otherwise}
\end{cases}.
\end{equation}
%
The properties
%
\begin{align}
T_{j + 2n}\parens{z_{k}^{n}} &= -T_{j}\parens{z_{k}^{n}} \nonumber \\ 
T_{j + 4n}\parens{z_{k}^{n}} &= T_{j}\parens{z_{k}^{n}}
\end{align}
%
follow from the definitions.
Taken together, we have the desired result.

When we have $2n+1$ columns, we see
%
\begin{equation}
    W = \begin{bmatrix}
            1 &   &   &   &   &   &   &   &   & -1 \\
              & 1 &   &   &   &   &   &   & -1& \\
             & &\ddots&   &   &   &  &\iddots&& \\
              &   &   &   & 1 & 0 & -1&   &   & \\
        \end{bmatrix}.
\end{equation}


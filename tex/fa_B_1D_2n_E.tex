\subsection{1D Birkhoff Interpolation on Chebyshev Nodes
and Endpoints with $2n$ Degree; First Factorization}
\label{ssec:fa_B_1D_2n_E_1}

In this section we present a fast algorithm for 1D interpolation
on the Chebyshev nodes using polynomials up to degree $2n$.
This is based on the $LQ$ factorization from Sec.~\ref{ssec:CV_D_1D_2n_E_F1}.

If we set
%
\begin{align}
    \begin{bmatrix} L & 0 \end{bmatrix}
        &= \begin{bmatrix} WD \\ \overline{A} \end{bmatrix} D_{s}^{-1}GH\Pi
        \nonumber\\
    \widehat{C} &= \begin{cases} C_{1} \quad n\text{ even} \\
        C_{0}\quad n\text{ odd} \end{cases},
\end{align}
%
noting that $L$ is the $n+2\times n+2$ lower triangular matrix
from Eqs.~\eqref{eq:CV_D_1D_2n_E_F1_L_even} or
\eqref{eq:CV_D_1D_2n_E_F1_L_odd}.
The algorithm is quite straight-forward:
%
\begin{align}
    y\parens{1\!:\!n+2} &= L^{-1}\begin{bmatrix} C^{-1}f \\ \widehat{C}^{-1}f
        \end{bmatrix} \nonumber\\
    a &= D_{s}^{-1}GH\Pi y.
\end{align}
%
%An efficient memory usage, one would only need one vector
%$y$ (initialized to zero) of length $2n+1$
%and the rest of the operations could overwrite the vector.



\subsection{1D Birkhoff Interpolation on Chebyshev Nodes
and Endpoints with $2n$ Degree; Second Factorization}
\label{ssec:fa_B_1D_2n_E_2}

In this section we present a fast algorithm for 1D interpolation
on the Chebyshev nodes using polynomials up to degree $2n$.
This is based on the $LQ$ factorization from Sec.~\ref{ssec:CV_D_1D_2n_E_F2}.

We set
%
\begin{align}
    \begin{bmatrix} L & 0 \end{bmatrix}
        &= \begin{bmatrix} UWD \\ \overline{A} \end{bmatrix} D_{s}^{-1}G\Pi
        \nonumber\\
    \widehat{C} &= \begin{cases} C_{1} \quad n\text{ even} \\
        C_{0}\quad n\text{ odd} \end{cases},
\end{align}
%
noting that $L$ is the $n+2\times n+2$ lower triangular matrix
from Eqs.~\eqref{eq:CV_D_1D_2n_E_F2_L_even} or
\eqref{eq:CV_D_1D_2n_E_F2_L_odd}.
The algorithm is quite straight-forward:
%
\begin{align}
    y\parens{1\!:\!n+2} &= L^{-1}\begin{bmatrix} UC^{-1}f \\ \widehat{C}^{-1}f
        \end{bmatrix} \nonumber\\
    a &= D_{s}^{-1}G\Pi y.
\end{align}
%
%An efficient memory usage, one would only need one vector
%$y$ (initialized to zero) of length $2n+1$
%and the rest of the operations could overwrite the vector.

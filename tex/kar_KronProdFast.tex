\section{Kronecker Products and Fast Matrix-Vector Multiplication}
\label{sec:FastKronProduct}

If $A$ and $B$ are structured matrices which allow for fast matrix-vector
multiplication, then intuitively it should be possible to compute
matrix-vector products involving $A\otimes B$ quickly as well.
In this section we demonstrate particular instances that are important
for the present research.
If $A\in\R^{m\times n}$, $B\in\R^{k\times\ell}$, and 
$g_{r} = g\parens{\brackets{r-1}\ell + 1\!:\!r\ell}$ with
$r\in\braces{1,\cdots,n}$, then, in block form, we have

\begin{equation}
    \parens{A\otimes B}g = \begin{bmatrix}
        a_{11}Bg_{1} + a_{12}Bg_{2} + \cdots + a_{1n}Bg_{n} \\
        a_{21}Bg_{1} + a_{22}Bg_{2} + \cdots + a_{2n}Bg_{n} \\ \vdots \\
        a_{m1}Bg_{1} + a_{m2}Bg_{2} + \cdots + a_{mn}Bg_{n} \\
    \end{bmatrix}.
\end{equation}

\noindent
If we set

\begin{equation}
    G = \begin{bmatrix} g_{1} & g_{2} & \cdots & g_{n} \end{bmatrix},
\end{equation}

\noindent
then we can compute the above matrix product like

\begin{samepage}
\begin{align}
    H &= BGA^{T} \nonumber\\
    &= \begin{bmatrix} h_{1} & h_{2} & \cdots & h_{m} \end{bmatrix},
\end{align}
\end{samepage}

\noindent
where $h_{r}$ has $k$ rows, and

\begin{align}
    \parens{A\otimes B}g &= h \nonumber\\
        &= \begin{bmatrix} h_{1} \\ h_{2} \\ \vdots \\ h_{m} \end{bmatrix}.
\end{align}

\noindent
It is this insight which allows us to compute matrix-vector multiplication
quickly when working with
tensor products of matrices, especially if
$A$ or $B$ allow for fast matrix-vector products.
We merely note that the above product (if $A$ and $B$ are square
matrices of size $n$) has cost $O(n^{4})$ flops if we explicitly
form $A\otimes B$.
By taking advantage of the tensor structure,
we can reduce the cost of multiplication to $O(n^{3})$ flops.
Naturally, this does not count memory transfer.

We will need to do something slightly different if we are performing
matrix-matrix multiplication. We see
%
\begin{align}
    \parens{A\otimes B}\begin{bmatrix} g_{1} & \cdots & g_{s} \end{bmatrix}
    &= \begin{bmatrix} \parens{A\otimes B}g_{1} & \cdots
        & \parens{A\otimes B}g_{s} \end{bmatrix} \nonumber\\
    &\sim \begin{bmatrix} BG_{1}A^{T} & \cdots & BG_{s}A^{T} \end{bmatrix}.
\end{align}
%
This could be computed quickly by
%
\begin{align}
    B\begin{bmatrix} G_{1} & \cdots & G_{s} \end{bmatrix}
    &= \begin{bmatrix} H_{1} & \cdots & H_{s} \end{bmatrix} \nonumber\\
    \begin{bmatrix} H_{1} \\ \vdots \\ H_{s} \end{bmatrix} A^{T}
    &= K,
\end{align}
%
before reading off the solution from the components of $K$.

Throughout this section, we assume that $F$ is a general matrix which
allows for fast matrix-vector products and we look at computing
$\parens{A\otimes F}g$ for a vector $g$ and particular matrix $A$.

For completeness, we recall some properties of the Kronecker product;
one reference is~\cite{vanLoan2000ubiquitous}.
These include bilinearity and associativity.
Furthermore,
%
\begin{equation}
    \parens{A\otimes B}\parens{C\otimes D} = AC\otimes BD.
\end{equation}
%
Here, we are assuming that the matrix products $AC$ and $BD$
are well-defined.
Additionally, there are permutation matrices (called perfect shuffle matrices)
$P$ and $Q$ so that
%
\begin{equation}
    P(A\otimes B)Q = B\otimes A.
\end{equation}



\subsection{Low-rank Matrices}
\label{ssec:FastKronProdLowRank}

We look at computing the product $\parens{wv^{*}\otimes F}g$,
where $w,v\in\R^{m}$ and $F\in\R^{n\times n}$.
First, we see
%
\begin{align}
    \parens{wv^{*}\otimes F}g &= \begin{bmatrix}
        w_{1}v_{1}F & w_{1}v_{2}F & \cdots & w_{1}v_{m}F \\
        w_{2}v_{1}F & w_{2}v_{2}F & \cdots & w_{2}v_{m}F \\
        \vdots & \vdots & \cdots & \vdots \\
        w_{m}v_{1}F & w_{m}v_{2}F & \cdots & w_{m}v_{m}F \\
    \end{bmatrix}
    \begin{bmatrix} g_{1} \\ g_{2} \\ \vdots \\ g_{m} \end{bmatrix} \nonumber\\
    &= \begin{bmatrix}
        w_{1}\parens{v_{1}Fg_{1} + v_{2}Fg_{2} + \cdots v_{m}Fg_{m}} \\
        w_{2}\parens{v_{1}Fg_{1} + v_{2}Fg_{2} + \cdots v_{m}Fg_{m}} \\
        \vdots \\
        w_{m}\parens{v_{1}Fg_{1} + v_{2}Fg_{2} + \cdots v_{m}Fg_{m}} \\
        \end{bmatrix} \nonumber\\
    &= w\otimes FGv,
\end{align}
%
where, as before, 
%
\begin{align}
    g_{k} &= g\parens{\brackets{k-1}n + 1\!:\!kn}
        \quad k\in\braces{1,\cdots,m} \nonumber\\
    G &= \begin{bmatrix}
        g_{1} & g_{2} & \cdots & g_{m}
        \end{bmatrix}.
\end{align}

Now, from the above computation, we see
%
\begin{align}
    \parens{WV^{*}\otimes F}g
        &= \parens{\begin{bmatrix} w_{1} & \cdots & w_{r} \end{bmatrix}
            \begin{bmatrix} v_{1} & \cdots & v_{r} \end{bmatrix}^{*}
            \otimes F}g \nonumber\\
        &= \parens{w_{1}v_{1}^{*} \otimes F}g + \cdots +
            \parens{w_{r}v_{r}^{*} \otimes F}g \nonumber\\
        &= w_{1}\otimes FGv_{1} + \cdots + w_{r}\otimes FGv_{r}
\end{align}
%
To efficiently compute this, we can use
%
\begin{align}
    H &= FGV \nonumber\\
        &= \begin{bmatrix} FGv_{1} & \cdots & FGv_{r} \end{bmatrix} \nonumber\\
        &= \begin{bmatrix} h_{1} & \cdots & h_{r} \end{bmatrix}
\end{align}
%
so that
%
\begin{equation}
    \parens{WV^{*}\otimes F}g = w_{1}\otimes h_{1} +
        \cdots + w_{r}\otimes h_{r}.
\end{equation}
%
Thus, we only need to compute fast multiplication with $F$ once.



\subsection{Householder Reflectors}
\label{ssec:FastKronProdHouse}

From Sec.~\ref{ssec:FastKronProdLowRank}, we can easily
compute $\parens{H\otimes F}g$, where $H = I - 2uu^{*}$ and
$u$ is a unit vector.
From properties of the Kronecker product, we see
%
\begin{align}
    \parens{H\otimes F}g
        &= \parens{I\otimes F}g - 2\parens{vv^{*}\otimes F}g \nonumber\\
        &= \parens{I\otimes F}g - 2\parens{v\otimes FGv}
\end{align}
%
We can quickly compute $\parens{I\otimes F}g$ from looking at the
columns of $FG$.



\subsection{Givens Rotations}
\label{ssec:FastKronProdGivens}

We assume our Givens rotation $P$ has the form
%
\begin{equation}
    P\parens{\brackets{i,j},\brackets{i,j}} =
        \begin{bmatrix} c & s \\ -s & c \end{bmatrix}.
\end{equation}
%
If $g_{k} = g\parens{\brackets{k-1}n + 1\!:\!kn}$ for
$k\in\braces{1,\cdots,m}$, then
%
\begin{equation}
    \parens{P\otimes F}g = \begin{bmatrix}
        Fg_{1} \\ \vdots \\ cFg_{i} + sFg_{j} \\ \vdots \\ -sFg_{i} + cFg_{j} \\
        \vdots \\ Fg_{m} \end{bmatrix},
\end{equation}
%
so that the $k$th block is $Fg_{k}$ except for blocks $i$ and $j$.

Frequently, we will have (possibly disjoint) products of Givens rotations.
We obtain similar results when applied to these products.

\section{Rotations and Structured Factorizations}
\label{sec:rotations}

Householder reflectors and Givens rotations are
standard orthogonal matrices which allow us to selectively
zero entries of a matrix.
Both will be important
in our matrix factorizations, for our fast algorithms require
us to compute the LQ factorization of Chebyshev-Vandermonde matrices.

\subsection{Householder Reflectors}
One special type of orthogonal matrix is a Householder reflector.
Given distinct nonzero $x$ and $y$ with the same length
(that is, $x\ne y$ and $\norm{x}_{2} = \norm{y}_{2} > 0$), we wish to find an
orthogonal matrix $P$ such that $Px = y$.
There is a reflection which does this: we set $v = x - y$ and choose
%
\begin{equation}
    P = I - \frac{2}{v^{*}v}vv^{*}.
\end{equation}
%
With this choice, we see
%
\begin{align}
    Px  &= x - \frac{2v^{*}x}{v^{*}v}v \nonumber\\
        &= x - \frac{2\parens{x^{*}x - x^{*}y}}{x^{*}x - 2x^{*}y + y^{*}y}
            \parens{x - y} \nonumber\\
        &= x - \frac{x^{*}x - 2x^{*}y + y^{*}y}{x^{*}x - 2x^{*}y + y^{*}y}
            \parens{x - y} \nonumber\\
        &= y
\end{align}
%
From the line 2 to line 3, we are using the fact that
$\norm{x}_{2} = \norm{y}_{2}$.
The previous work also holds when
$x,y\in\C^{n}\setminus\braces{0}$.

We frequently want to choose a vector parallel to $e_{1}$,
allowing us to zero most of the entries of $x$.
In this case, we choose $v = x - \norm{x}_{2}e_{1}$.
One difficulty when $v$ is computed numerically is
the catastrophic cancellation that can occur in $v_{1}$.
Care must be taken in order to ensure this does not happen;
see Alg.~\ref{alg:unit_house}, which has been modified
from the version in~\cite[Alg.~5.1.1]{gvl4}.

\begin{algorithm}[t]
\caption{Householder Reflector}
\label{alg:unit_house}
\begin{algorithmic}[1]
\Function{house}{$x$}
\Comment{Numerically stable way to compute Householder Reflector}
\State $m = \text{length}(x)$
\State $v = \text{zeros}(m)$
\State $\sigma = \norm{x(2:m)}_{2}^{2}$
\State $\mu = \sqrt{x_{1}^{2} + \sigma}$
\If{$x_{1}\le0$}
    \State $v_{1} = x_{1} - \mu$
\Else
    \State $v_{1} = -\frac{\sigma}{x_{1} + \mu}$
\EndIf
\State $v = v/\norm{v}_{2}$
\State \Return $v$
\EndFunction
\end{algorithmic}
\end{algorithm}


Matrix-vector products involving Householder reflector $P$ can be
computed quickly by taking advantage of its structure; it can
be computed and applied in $O(n)$ flops.
This implies that a small, constant number of reflectors
can be applied with total cost $O(n)$.
Also, in practice we never need to explicitly store $P$, only $v$.
In this work, we will be assuming that our reflection vector
$v$ has unit length.



\subsection{Givens Rotations}
We make use of Givens rotations, so we review them and their
ability to selectively zero entries of a matrix. This is important
because our matrices are structured.
A Givens rotation $G$ is the identity matrix with a rank-2 correction:
%
\begin{equation}
    G\parens{\brackets{i,j},\brackets{i,j}}
        = \begin{bmatrix} c & s \\ -s & c \end{bmatrix}.
\end{equation}
%
Here, $c = \cos\theta$ and $s = \sin\theta$ for some $\theta$.
We do not need to determine $\theta$ explicitly in practice,
for we require
%
\begin{equation}
    \begin{bmatrix}\alpha & \beta \end{bmatrix}
    \begin{bmatrix} c & s \\ -s & c \end{bmatrix}
        = \begin{bmatrix} r & 0 \end{bmatrix}.
    \label{eq:givens_def}
\end{equation}
%
Thus, we are finished if we can write $c$ and $s$ in terms of
$\alpha$ and $\beta$.
This is easy, though, and a numerically stable
way to compute $c$ and $s$ is given in Alg.~\ref{alg:givens}.
Our convention here differs from others
(such as \cite[Alg.~5.1.3]{gvl4} or \cite{bindel2002computing})
by the fact that we ensure $r = \sqrt{\alpha^{2}+\beta^{2}}$
in Eq.~\eqref{eq:givens_def}.
The total cost for computing a Givens rotation and
applying it to a vector is $O(1)$.
This allows us to compute and apply $O(n)$ Givens rotations for $O(n)$
total cost.

\begin{algorithm}
\caption{Givens Rotation}
\label{alg:givens}
\begin{algorithmic}[1]
\Function{givens}{$\alpha$,$\beta$}
\Comment{Numerically stable way to compute Givens rotations}
    \If{$\abs{\alpha}\ge\abs{\beta}$}
        \If{$\alpha=0$}
            \State $c=1$
            \State $s = 0$
        \ElsIf{$\beta=0$}
            \State $c = \sign(\alpha)$
            \State $s = 0$
        \Else
            \State $\tau = -\frac{\beta}{\alpha}$
            \State $c = \sign(\alpha)/\sqrt{1 + \tau^{2}}$
            \State $s = c\tau$
        \EndIf
    \Else
        \If{$\alpha = 0$}
            \State $s = -\sign(\beta)$
            \State $c = 0$
        \Else
            \State $\tau = -\frac{\alpha}{\beta}$
            \State $s = -\sign(\beta)/\sqrt{1 + \tau^{2}}$
            \State $c = s\tau$
        \EndIf
    \EndIf
\State \Return $c$, $s$
\EndFunction
\end{algorithmic}
\end{algorithm}


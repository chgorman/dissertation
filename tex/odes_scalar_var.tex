\section{Variable-Coefficient Scalar ODE}
\label{sec:odes_scalar_var}

When we have the differential equation
%
\begin{align}
    \alpha(x)u'(x) + \beta(x)u(x) &= f(x) \nonumber\\
    Lu &= g,
\end{align}
%
the cost of solving the linear system becomes higher.
The good news is that much remains the same; the primary difference
comes in the off-diagonal ranks of the $H_{i}$ matrices.
If we have
%
\begin{align}
    \alpha(x) &= \sum_{k=0}^{r} a_{k}T_{k}(x) \nonumber\\
    \beta(x)  &= \sum_{k=0}^{r} b_{k}T_{k}(x),
\end{align}
%
then from the results in Sec.~\ref{sec:CV_lin_comb} show that
now $H_{1}$ and $H_{2}$ will be ``banded'' matrices with bandwidth $r+1$
(technically, $H_{2}$ is banded along the antidiagonal);
thus, they will have off-diagonal rank of $r+1$ while
$H_{3}$ will have off-diagonal rank $2r+2$.
The fast algorithms will still apply, save the fact that now
the structured solver will take more time.
The boundary condition still adds one additional linear constraint
to the overall system and is not complicated by the variable $\alpha$
and $\beta$.

If the coefficients are not polynomials, then the infinite series
can be truncated.
In the case of smooth $\alpha$ and $\beta$, only a few terms should
be required until errors from machine precision take over.
Bounds similar to those in Sec.~\ref{sec:odes_h13_cond} for the
constant coefficient case should be possible so long as $\alpha_{0}$
is sufficiently large to ensure diagonal dominance.
Regardless, the fast algorithm still applies with cost proportional
to the off-diagonal rank.


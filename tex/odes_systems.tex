\section{Systems of ODEs}
\label{sec:odes_systems}

We are now ready to look at systems of ODEs and will
start with the constant coefficient case:
%
\begin{align}
    Au'(x) + Bu(x) &= F(x) \nonumber\\
    Lu = G.
\end{align}
%
Here, we are seeking a solution
%
\begin{align}
    u(x) &= \begin{bmatrix}
            u_{1}(x) \\
            u_{2}(x) \\
            \vdots \\
            u_{m}(x) \\ \end{bmatrix} \nonumber\\
    u_{\ell}(x) &= \sum_{k=0}^{2n} a_{\ell;k}T_{k}(x).
\end{align}
%
with the coefficients arranged like
%
\begin{equation}
    a^{*} = \begin{bmatrix}
        a_{1;0} & a_{1;1} & \cdots & a_{1;2n} &
        a_{2;0} & a_{2;1} & \cdots & a_{2;2n} & \cdots
        a_{m;0} & a_{m;1} & \cdots & a_{m;2n} \end{bmatrix}
\end{equation}
%
From this arrangement, we see the results linear system will be
%
\begin{equation}
    \brackets{A\otimes VD + B\otimes V}a = F,
\end{equation}
%
where $F$ will store all the values from the righthand side.
After applying the tensored IDCT $I\otimes C^{-1}$
gives
%
\begin{equation}
    \brackets{A\otimes WD + B\otimes W}a = \widehat{F}.
\end{equation}

\noindent
Applying perfect shuffle matrices~\cite{vanLoan2000ubiquitous} $P$
and $Q$, we can switch the order of the Kronecker product:
%
\begin{equation}
    P\brackets{A\otimes WD + B\otimes W}QQ^{*}a = P\widehat{F},
\end{equation}
%
which implies
%
\begin{equation}
    \brackets{WD\otimes A + W\otimes B}Q^{*}a = P\widehat{F}.
\end{equation}
%
Multiplying by $U\otimes I$ on the left gives
%
\begin{equation}
    \brackets{UWD\otimes A + UW\otimes B}Q^{*}a
        = \parens{U\otimes I}P\widehat{F}.
\end{equation}
%
If $\Pi$ is the circular downshift permutation matrix, then
%
\begin{equation}
    \brackets{UWD\otimes A + UW\otimes B}\parens{\Pi\otimes I}
    = \begin{bmatrix} H_{1} & H_{2} \end{bmatrix}.
\end{equation}

In this case, we have the following block matrix:
%
\begin{equation}
    H_{1} = \begin{bmatrix}
        A               & -\frac{1}{2}B         & \\
        \frac{1}{2}B & 2A & -\frac{1}{2}B \\
        & \frac{1}{2}B & 3A & -\frac{1}{2}B \\
        & & \ddots & \ddots & \ddots \\
        & & & \frac{1}{2}B & (n-2)A & -\frac{1}{2}B \\
        & & & & \frac{1}{2}B & (n-1)A & 0 \\ 
        & & & & & \frac{1}{2}B & nA \\
    \end{bmatrix}.
\end{equation}
%
This matrix has the exact same form as the constant coefficient
example in Eq.~\eqref{eq:odes_h1_mat}.
This time, $H_{1}$ now has off-diagonal blocks for rank $m$,
implying $H_{1}$ (and $H_{2}$) are matrices with SSS and HSS rank $m$
and that $H_{3}$ has rank $2m$.

By performing similar operations in the variable-coefficient
scalar case, we see that if
%
\begin{align}
    A(x) &= \sum_{k=0}^{r} A_{k}T_{k}(x) \nonumber\\
    B(x) &= \sum_{k=0}^{r} B_{k}T_{k}(x),
    \label{eq:odes_systems_coefs}
\end{align}
%
then the $H_{1}$ and $H_{2}$ matrices will have HSS rank $m(r+1)$
and $H_{3}$ will have rank $2m(r+1)$.




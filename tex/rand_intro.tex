\section{Randomized Low-Rank Approximation}
\label{sec:rand_lra}

In recent years there has been growing interest in using randomization
to speed up conventional numerical linear algebra techniques;
one standard reference is~\cite{RandomReview2011}.
The main idea is that we would like to compute
accurate factorizations like pivoted QR or the SVD on low-rank matrices.
These factorizations usually have  flop counts of
$O(mnr)$ and $O(mn^{2})$, respectively,
but the data transfer required makes these computations expensive,
especially when matrices are so large as to not fit in fast memory.
Thus, it is impractical to use standard dense algorithms with
large communication costs on distributed memory machines.
By looking at random samplings of the range, the desire is
to build up an approximation of the matrix until reaching
a predetermined approximation tolerance.
Once a sufficient number of random samples have been computed,
we perform a pivoted QR or SVD on this smaller dense
matrix, leading to a smaller overall cost.
The random samplings are computed using matrix-matrix multiplication,
which can be efficiently computed on modern hardware.
Although these multiplications may be a significant portion
of the overall flop count, the total time
spent performing multiplication is small.
Thus, the goal is to determine when enough samples have been computed
to ensure an accurate approximation of the matrix.
The standard algorithm for building a low-rank approximation
is found in Alg.~\ref{alg:rand_low_rank}, with the primary
arguments being the block size $d$ and \texttt{stopping\_criterion},
a function which determines when we have a good enough approximation
of the range of $A$; similar blocked algorithms can be found
in~\cite{yu2018efficient,martinsson2016randomized} and are reproduced below.
In Line~\ref{alg_line:rand_Gen_GS} of Alg.~\ref{alg:rand_low_rank},
we perform iterated Gram-Schmidt orthogonalization for numerical stability,
as it helps ensure $Q$ is \emph{numerically} orthogonal to machine
precision, especially in the blocked case~\cite{bjorck1994GS,stewart2008}.
This may not always be necessary, for \cite{stewart2008} notes
this would only be required should the column norms decrease by a significant
factor; even so, algorithmically it is easier, though more expensive,
to always perform iterated GS.
Additionally, although \texttt{rand} could produce any kind of random
matrices, our analysis requires us to use Gaussian random variables.
See Table~\ref{tab:rand_helper_funcs} for a list of helper functions
for the algorithms in this chapter.
We denote the pivoted QR factorization as \texttt{rrqr}
and it stops when $\abs{R_{k,k}}<\epsa$ or
$\abs{R_{k,k}} < \epsr\abs{R_{1,1}}$.

\begin{table}
\centering
\begin{tabular}{r|p{0.7\textwidth}}
\hline
\texttt{rand}(m,n) & an $m\times n$ matrix with iid $N(0,1)$ elements \\
\texttt{cols}(A) & number of columns of $A$ \\
$\texttt{qr}(A)$ & unpivoted QR factorization of $A$ returning $Q$
    or $Q$ and $R$ \\
$\texttt{rrqr}(A,\epsa,\epsr)$ & rank-revealing QR factorization of $A$
    such as QRCP with absolute/relative tolerances $\epsa$/$\epsr$ \\
\hline
\end{tabular}
\caption[List of helper functions for low-rank approximation]{
List of helper functions for low-rank approximation.}
\label{tab:rand_helper_funcs}
\end{table}


These algorithms frequently use a QB approximation of $A$.
Here $Q$ is an approximate orthonormal basis for the range of $A$
with $B = Q^{*}A$; that is
%
\begin{align}
    A &\approx Q(Q^{*}A) \nonumber\\
    &= QB.
\end{align}
%
Knowing when $\norm{A - QB}$ is small is our primary concern,
and the particular choice of norm is important.
This chapter will investigate when this happens,
propose a new method, and compare it with existing stopping criteria.
The stopping criterion presented here has focused on developing
a \emph{relative} stopping criterion.
This is particularly useful as it may be more convenient
to specify a relative error tolerance over an absolute tolerance.
Furthermore, we would like this method to work in the case
we do not have explicit access to the matrix.
Theorem 4.2 in~\cite{vogel2016superfast} says that if $H$ is a
structured $N\times N$ matrix and $\widetilde{H}$ is $H$
with off-diagonal blocks truncated
to tolerance $\epsr$, then
%
\begin{equation}
    ||H - \widetilde{H}||_{F} \le C(N,L)\epsr\norm{H}_{F}.
\end{equation}
%
Thus, it makes sense to seek an accurate relative stopping criterion.
Once a sufficient number of random samples have been computed,
we perform a pivoted QR factorization
(such as QR with Column Pivoting, although the Strong RRQR
from~\cite{gu1996efficient} would be better) on these samples
in order to obtain a high-quality $Q$ for a better QB approximation.
While this last step may not be necessary and was not included
in~\cite{yu2018efficient,martinsson2016randomized},
this would be an easy step to add and was desired in
the original setting where this stopping criterion was
developed~\cite{randomHSSLBL}.

\begin{algorithm}[t]
\caption{Randomized Block Low-Rank Approximation (General)}
\label{alg:rand_low_rank}
\begin{algorithmic}[1]
\Function{random\_low\_rank}{$A$,$d$,\texttt{stopping\_criterion}}
\Comment{Builds $QB$ approximation}
    \State $Q = [\,]$; $n = \texttt{cols}(A)$
    \State $k=0$
    \While {!\texttt{stopping\_criterion}($Q$,$A$,$\eps$)}
        \State $k = k + 1$
        \State $\Omega_{k} = \texttt{rand}(n,d)$
        \Comment{$d$ is block size}
        \State $S_{k} = A\Omega_{k}$
        \State $\widehat{S}_{k} = \parens{I - QQ^{*}}^{2}S_{k}$
        \Comment{$2\times$ GS orthogonalization for numerical stability}
        \label{alg_line:rand_Gen_GS}
        \State $[Q_{k},R_{k}] = \texttt{qr}(\widehat{S}_{k})$
        \State $Q = \begin{bmatrix}Q & Q_{k}\end{bmatrix}$
    \EndWhile
\State $B = Q^{*}A$
\State \Return $Q$, $B$
\Comment{We have the approximation $A \approx QB$}
\EndFunction
\end{algorithmic}
\end{algorithm}


We recall the Eckhart-Young theorem~\cite[Theorem~1.1]{gu2015subspace},
which gives optimal low-rank approximation bounds in the 2-norm and F-norm:

\begin{thm}[Eckhart-Young Theorem]
If $A_{k}$ is the matrix $A$ chopped to $k$ singular values,
then we have the following bounds on approximations of $A$:
%
\begin{align}
    \min_{\text{rank}(B)\le k}\norm{A-B}_{2} &= \norm{A-A_{k}}_{2} \nonumber\\
        &= \sigma_{k+1} \nonumber\\
    \min_{\text{rank}(B)\le k}\norm{A-B}_{F} &= \norm{A-A_{k}}_{F} \nonumber\\
        &= \sqrt{\sum_{j=k+1}^{\min(m,n)}\sigma_{j}^{2}}.
\end{align}
\end{thm}

\noindent
In practice, we hope to be able to determine approximations
with near-optimal ranks within a specified tolerance
with smaller overall computation time.
Thus, we are looking at the fixed-precision low-rank approximation problem.



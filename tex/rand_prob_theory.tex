\section{Basic Probability Theory}
\label{sec:rand_prob_theory_intro}

We assume $A$ has positive singular values
$\sigma_{1}\ge\cdots\ge\sigma_{r}>0$.
It follows that we have the SVD
%
\begin{align}
    A &= U\Sigma V^{*} \nonumber\\
    &= \begin{bmatrix} U_{1} & U_{2} \end{bmatrix}
        \begin{bmatrix} \Sigma_{r} & 0 \\ 0 & 0 \end{bmatrix}
        \begin{bmatrix} V_{1}^{*} \\ V_{2}^{*} \end{bmatrix} \nonumber\\
    &= U_{1}\Sigma_{r}V_{1}^{*},
\end{align}
%
where
%
\begin{equation}
    \Sigma_{r} = \diag\parens{\sigma_{1},\cdots,\sigma_{r}}.
\end{equation}
%
If $x\in\R^{n}$ is a Gaussian random vector (that is, $x_{i}\sim N(0,1)$)
and we set $\xi = V^{*}x$, then by the rotational invariance
of $\norm{\cdot}_{2}$ we have
%
\begin{align}
    \norm{Ax}_{2}^{2} &= \norm{\Sigma_{r}\xi}_{2}^{2} \nonumber\\
        &= \sigma_{1}^{2}\xi_{1}^{2} + \cdots + \sigma_{r}^{2}\xi_{r}^{2}.
\end{align}
%
Now, $\xi_{i}\sim N(0,1)$ because rotations of Gaussian random vectors
are also Gaussian random vectors.
From here, we can see
%
\begin{equation}
    \E\norm{Ax}_{2}^{2} = \norm{A}_{F}^{2}.
\end{equation}
%
This by itself is useful, but, in order to make use of
computer architecture, matrix-matrix products are preferred over
multiple matrix-vector products.
Thus, if $\Omega\in\R^{n\times d}$ with $\Omega_{ij}\sim N(0,1)$
(so that $\Omega$ is a Gaussian random matrix), it follows that
%
\begin{equation}
    \norm{A\Omega}_{F}^{2} = \norm{A\Omega_{:,1}}_{2}^{2} + \cdots
        + \norm{A\Omega_{:,d}}_{2}^{2}.
\end{equation}
%
Because each $\Omega_{:,i}$ is a Gaussian random vector, we also have
%
\begin{equation}
    \E\norm{A\Omega}_{F}^{2} = d\norm{A}_{F}^{2}.
    \label{eq:rand_new_error_bound}
\end{equation}
%
It is this equality that allows us to accurately compute the F-norm
of matrices using random range samples.

Because there is interest in using power iterations to increase
the quality of randomized low-rank factorizations~\cite{RandomReview2011},
we also include these related results.
In particular, it is clear from the previous work that
%
\begin{align}
    \norm{\parens{AA^{*}}^{q}Ax}_{2}^{2}
        &= \sigma_{1}^{4q+2}\xi_{1}^{2} + \cdots + \sigma_{r}^{4q+2}\xi_{r}^{2}
        \nonumber\\
    \norm{\parens{A^{*}A}^{q}x}_{2}^{2}
        &= \sigma_{1}^{4q}\xi_{1}^{2} + \cdots + \sigma_{r}^{4q}\xi_{r}^{2},
\end{align}
%
so we have
%
\begin{align}
    \E\norm{\parens{AA^{*}}^{q}Ax}_{2}^{2}
        &= \norm{A}_{s,4q+2}^{4q+2}
        \nonumber\\
    \E\norm{\parens{A^{*}A}^{q}x}_{2}^{2}
        &= \norm{A}_{s,4q}^{4q}
\end{align}
%
and
%
\begin{align}
    \E\norm{\parens{AA^{*}}^{q}A\Omega}_{2}^{2}
        &= d\norm{A}_{s,4q+2}^{4q+2}
        \nonumber\\
    \E\norm{\parens{A^{*}A}^{q}\Omega}_{2}^{2}
        &= d\norm{A}_{s,4q}^{4q}.
    \label{eq:rand_new_power_error_bound}
\end{align}
%
Here, $\norm{\cdot}_{s,p}$ is the Schatten $p$-norm defined
in Eq.~\eqref{eq:kar_schatten_def} and $\Omega$ is a Gaussian random
matrix with $d$ columns, as before.
From the definitions of the Schatten $p$-norm, it is clear
%
\begin{align}
    \brackets{\E\norm{\parens{AA^{*}}^{q}Ax}_{2}^{2}}^{1/4q+2}
        &\to \norm{A}_{2}
        \nonumber\\
    \brackets{\E\norm{\parens{A^{*}A}^{q}x}_{2}^{2}}^{1/4q}
        &\to \norm{A}_{2}
        \nonumber\\
    \brackets{\E\norm{\parens{AA^{*}}^{q}A\Omega}_{2}^{2}}^{1/4q+2}
        &\to \norm{A}_{2}
        \nonumber\\
    \brackets{\E\norm{\parens{A^{*}A}^{q}\Omega}_{2}^{2}}^{1/4q}
        &\to \norm{A}_{2}
\end{align}
%
as $q\to\infty$.

The proof that independent realizations of Eqs.~\eqref{eq:rand_new_error_bound}
and \eqref{eq:rand_new_power_error_bound}
produce accurate approximations of the F-norm and Schatten norm
will be postponed until Sec.~\ref{sec:rand_prob_theory_proofs}.
We now focus on developing our stopping criterion, which is based on
the fact that we can now accurately estimate our error in a matrix norm.





\section{New Stopping Criterion}
\label{sec:rand_new_stopping}

We now step through Alg.~\ref{alg:rand_low_rank} to see
where some problems, previously ignored, may arise;
this stems from the fact that we are wanting to be able to have
relative error tolerances on the order of $10^{-12}$ or $10^{-14}$
in double precision.
We will be building our random matrices up in blocks of $d$ random
vectors at a time, although in theory the size of the blocks could vary.
Assume we have our absolute tolerance $\epsa$ and relative tolerance $\epsr$.
Then our \texttt{stopping\_criterion} function will return \texttt{true} if
%
\begin{equation}
    ||\widehat{S}_{k}||_{F} < \max(\epsa\sqrt{d},\epsr\norm{S_{k}}_{F});
\end{equation}
%
otherwise, \texttt{stopping\_criterion} returns \texttt{false}.

\begin{enumerate}

\item Set $\Omega_{i} = \texttt{rand}(n,d)$ and compute $S_{i} = A\Omega_{i}$
for $i\in\braces{1,2}$; set $Q = [\,]$ and
$S = \begin{bmatrix} S_{1} & S_{2} \end{bmatrix}$.

\item Compute $[Q_{1},R_{1}] = \texttt{qr}(S_{1})$,
giving the initial approximation for the range of $A$; set $Q = Q_{1}$.

\item Set $\widehat{S}_{2} = \parens{I-QQ^{*}}^{2}S_{2}$, so that
$\widehat{S}_{2}$ contains potentially new information
about the range of $A$.
If
%
\begin{equation}
    ||\widehat{S}_{2}||_{F} < \max(\epsa\sqrt{d},\epsr||S_{2}||_{F}),
\end{equation}
%
then stop and perform $Q = \texttt{rrqr}(S,\epsa,\epsr)$.

\item Set $\Omega_{3} = \texttt{rand}(n,d)$ and compute $S_{3} = A\Omega_{3}$;
set $S = \begin{bmatrix} S & S_{3} \end{bmatrix}$.

\item Compute $[Q_{2},R_{2}] = \texttt{qr}(\widehat{S}_{2})$,
giving new information for the range of $A$;
set $Q = \begin{bmatrix} Q & Q_{2} \end{bmatrix}$.

\item Set $\widehat{S}_{3} = \parens{I-QQ^{*}}^{2}S_{3}$, so that
$\widehat{S}_{3}$ contains potentially new information
about the range of $A$.
If
%
\begin{equation}
    ||\widehat{S}_{3}||_{F} < \max(\epsa\sqrt{d},\epsr||S_{3}||_{F}),
\end{equation}
%
then stop and perform $Q = \texttt{rrqr}(S,\epsa,\epsr)$.

\item Continue this process until convergence $\dots$

\end{enumerate}

\noindent
Using individual realizations of Eq.~\eqref{eq:rand_new_error_bound}
is important in the stopping criterion, although the relative stopping
$||\widehat{S}_{k}||_{F} < \epsr||S_{k}||_{F}$ is new.
Even so, in order to ensure we are adding new information to $Q$ in steps
2 and 5, we need to know $\widehat{S}_{k}$ is full rank.
This problem does not show up in the unblocked version because
updating $Q$ one vector at a time makes it easy to determine
when a vector has small norm.
This situation, when $\widehat{S}_{k}$ is numerically low-rank,
is not considered in
Algs.~\ref{alg:rand_low_rank_MV} and \ref{alg:rand_low_rank_YGL}
due to how the $Q_{k}$ blocks are formed,
although this is critical to ensure $Q$ has orthonormal columns.
As noted previously, performing multiple matrix-vector multiplications
is inefficient on modern machines, which is why the blocked case is important.
Thus, if $\widehat{S}_{k}$ is (numerically) rank-deficient,
then additional random samples contain no new information above the
specified tolerance and we should
compute the rank-revealing QR factorization on our random samples.
If $\widehat{S}_{k}$ is determined to be rank-deficient, then
$\begin{bmatrix} S_{1} & S_{2} & \cdots & S_{k} \end{bmatrix}$
is rank-deficient.
This implies the random samples matrix
%
\begin{equation}
    S = \begin{bmatrix} S_{1} & S_{2} & \cdots & S_{k} & S_{k+1} \end{bmatrix}.
\end{equation}
%
is rank-deficient as well and the additional block of random samples $S_{k+1}$
should ensure a better quality RRQR factorization than just using
$\begin{bmatrix} S_{1} & S_{2} & \cdots & S_{k} \end{bmatrix}$.
Although this is not a proof, extensive examples
in Sec.~\ref{sec:rand_exp} show that this is expected
and matches our intuition.

To determine numerical rank-deficiency, we set
%
\begin{equation}
    \rho = \frac{||S_{1}||_{F}}{\sqrt{d}},
\end{equation}
%
implying $\rho\approx\norm{A}_{F}$,
and say $\widehat{S}_{k}$ is numerically rank-deficient when
%
\begin{equation}
    \min_{j=1,\cdots,d} |\parens{R_{k}}_{j,j}| < \max(\epsa,\epsr\rho).
\end{equation}
%
The purpose of $\rho$ is to characterize how large the original norms
of the random samples should be, as this will help determine when
they have fallen below the desired tolerance.
Because of this, other potential values of $\rho$ are $|(R_{1})_{1,1}|$,
$\max_{j} |(R_{1})_{j,j}|$, or $\max_{j,k} |(R_{k})_{j,j}|$.
If the desire is to minimize communication, then
$|(R_{1})_{1,1}|$ or $\max_{j} |(R_{1})_{j,j}|$ may be preferred.
Because \texttt{qr} is unpivoted, there is no guarantee the diagonals
of $R$ will always decrease in value.
Even so, extensive tests have shown there is general decay, but more
theoretical work that needs to be done.
This is one reason for using the additional block of
samples $S_{k+1}$ in \texttt{rrqr}.

Taken together, this gives us a new stopping criterion for low-rank
approximations, presented in Alg.~\ref{alg:rand_new_low_rank}.
At this point, we now compare our stopping criterion with the ones
previously mentioned.

\begin{algorithm}[t]
\caption{Randomized Block Low-Rank Approximation (New)}
\label{alg:rand_new_low_rank}
\begin{algorithmic}[1]
\Function{random\_new\_low\_rank}{$A$,$d$,$\epsa$,$\epsr$}
\Comment{Builds $QB$ approximation}
    \State $Q = [\,]$; $n = \texttt{cols}(A)$
    \State $\Omega_{i} = \texttt{rand}(n,d)$, $S_{i} = A\Omega_{i}$
        for $i\in\braces{1,2}$
    \Comment{$d$ is block size}
    \State $S = \begin{bmatrix} S_{1} & S_{2} \end{bmatrix}$
    \State $[Q_{1},R_{1}] = \texttt{qr}(S_{1})$; $Q = Q_{1}$
    \State $\rho = \norm{S_{1}}_{F}/\sqrt{d}$
    \If {$\min_{j} |(R_{1})_{jj}| < \max(\epsa,\epsr\rho)$}
        \State $\texttt{loop\_bool} = \texttt{false}$
        \Comment{$S_{1}$ is numerically rank-deficient}
    \Else
        \State $\texttt{loop\_bool} = \texttt{true}$
    \EndIf
    \State $k=1$
    \While {\texttt{loop\_bool}}
        \State $k = k + 1$
        \State $\widehat{S}_{k} = \parens{I - QQ^{*}}^{2}S_{k}$
        \Comment{$2\times$ GS orthogonalization for numerical stability}
        \label{alg_line:rand_new_GS_2x}
        \If {$||\widehat{S}_{k}||_{F} < \max(\epsa\sqrt{d},\epsr||S_{k}||_{F})$}
            \State \textbf{break}
            \Comment{Range of $A$ is known to specified tolerance}
        \EndIf
        \State $\Omega_{k+1} = \texttt{rand}(n,d)$; $S_{k+1} = A\Omega_{k+1}$
        \State $S = \begin{bmatrix} S & S_{k+1} \end{bmatrix}$
        \State $[Q_{k},R_{k}] = \texttt{qr}(\widehat{S}_{k})$
        \If {$\min_{j} |(R_{k})_{jj}| < \max(\epsa,\epsr\rho)$}
            \State \textbf{break}
            \Comment{$\widehat{S}_{k}$ is numerically rank-deficient}
        \EndIf
        \State $Q = \begin{bmatrix}Q & Q_{k}\end{bmatrix}$
    \EndWhile
\State $Q = \texttt{rrqr}(S,\epsa,\epsr)$
\State $B = Q^{*}A$
\State \Return $Q$, $B$
\Comment{We have the approximation $A \approx QB$}
\EndFunction
\end{algorithmic}
\end{algorithm}


The MV stopping criterion explicitly assumes an absolute stopping tolerance.
The YGL stopping criterion has an absolute stopping tolerance that
requires and depends on $\norm{A}_{F}$, so this could be viewed
as a relative stopping criterion with the restriction
that $\epsr \ge C\sqrt{\epsm}$.
One important benefit to the MV and YGL stopping criteria
is that they are exact: we know the exact error in the F-norm
at each step in the process.
Our new stopping criterion allows for absolute and relative
stopping criteria explicitly; furthermore, the relative tolerance is
limited by the ability to accurately compute
%
\begin{equation}
    ||\widehat{S}_{k}||_{F} < \epsr\norm{S_{k}}_{F}
    \quad\text{and}\quad \min_{j=1,\cdots,d} |(R_{k})_{j,j}| < \epsr\rho.
\end{equation}
%
Thus, we expect to be able to compute relative tolerances down to
$O(\eps_{\text{mach}})$, although we have not determined the exact
restrictions.
It is likely that we must have $\epsr\ge C_{n,d}\epsm$,
but we will not investigate the nature of $C_{n,d}$ at this
time, as it is related to the accuracy of matrix-matrix
multiplication~\cite[Chapter 3]{HighamASNA},
blocked GS~\cite{bjorck1994GS,stewart2008},
and QR factorizations~\cite[Chapter 19]{HighamASNA}.
The results in Sec.~\ref{sec:rand_exp} and~\cite{randomHSSLBL}
show that we can obtain excellent
results using our relative stopping criterion
even though our estimates are probabilistic.



\section{Probability Theory Proofs}
\label{sec:rand_prob_theory_proofs}

To simplify our analysis, we begin by defining a random variable
which has the same properties as $\norm{Ax}_{F}^{2}$:
%
\begin{equation}
    X \sim \sigma_{1}^{2}\xi_{1}^{2} + \cdots + \sigma_{r}^{2}\xi_{r}^{2}.
\end{equation}
%
Here, $\xi_{i}\sim N(0,1)$,
so we have $\E\norm{Ax}_{2}^{2} = \norm{A}_{F}^{2}$.
We now average $X$ to arrive at the random variable
%
\begin{equation}
    \overline{X}_{d} \sim \frac{1}{d}\brackets{X_{1} + \cdots + X_{d}}.
    \label{eq:rand_averaged_RV}
\end{equation}
%
$X_{i}$ are independent and identically distributed realizations of $X$.
Obviously, we also have $\E(\overline{X}_{d}) = \norm{A}_{F}^{2}$ as well as
%
\begin{align}
    \V(\overline{X}_{d}) &= \frac{1}{d}\V(X) \nonumber\\
        &= \frac{2\norm{A}_{s,4}^{4}}{d}.
    \label{eq:rand_Xd_variance_bounds}
\end{align}
%
The above result is well-known about the variance of independent random
variables.

We now seek a concentration inequality
for $\overline{X}_{d}$.
To do this, we need Chernoff's Inequality~\cite{prob_book},
which is useful for bounding tail probabilities:

\begin{thm}[Chernoff's Inequality;
    Theorem 3.2.2 in~\cite{prob_book}]\label{thm:Chernoff}
Given a random variable $X$, we have
%
\begin{equation}
    \P\brackets{X \ge a} \le \min_{t > 0} e^{-ta}\, \E\parens{e^{tX}}.
    \label{eq:Chernoff}
\end{equation}
%
and
%
\begin{equation}
    \P\brackets{X \le a} \le \min_{t>0} e^{ta}\,\E\parens{e^{-tX}}.
\end{equation}
\end{thm}

\noindent
Here, $\E\parens{e^{tX}}$ is the moment generating function
for the random variable $X$.
We will use these inequalities to prove the tail probabilities
of $\overline{X}_{d}$ decay exponentially in $d$,
ensuring independent realizations of
Eq.~\eqref{eq:rand_new_error_bound} are close to the expected value.
This result and proof were published
in~\cite{randomHSSLBL} but we flesh out some of the details not
included due to length considerations.



\begin{thm}[Probabilistic Error Bounds]\label{thm:prob_err_bound}
Given $\overline{X}_{d}$ as defined in Eq.~\eqref{eq:rand_averaged_RV}
with $r\ge2$, the following bounds on the tail probabilities hold:
%
\begin{align}
    \P\brackets{\overline{X}_{d}\ge \norm{A}_{F}^{2}\mu}
        &\le \exp\parens{-\frac{d\mu}{2}} \norm{A}_{F}^{dr}
        \prod_{k=1}^{r}\parens{A_{k}'}^{-d}
        \quad \mu>1 \nonumber\\
    \P\brackets{\overline{X}_{d}\le \norm{A}_{F}^{2}\mu}
    &\le \exp\parens{\frac{d\mu}{2}} \norm{A}_{F}^{dr}
        \prod_{k=1}^{r}\parens{A_{k}''}^{-d}
        \quad \mu\in[0,1).
\end{align}
%
Here,
%
\begin{align}
    \norm{A}_{F}^{2} &= \sigma_{1}^{2} + \cdots + \sigma_{r}^{2} \nonumber\\
    \parens{A_{k}'}^{2} &= \norm{A}_{F}^{2} - \sigma_{k}^{2} \nonumber\\
    \parens{A_{k}''}^{2} &= \norm{A}_{F}^{2} + \sigma_{k}^{2}.
    \label{eq:rand_As_def}
\end{align}
%
We know $\E\parens{\overline{X}_{d}} = \norm{A}_{F}^{2}$, so $\mu$
controls multiplicative deviation above or below the expectation value.
Furthermore, if
%
\begin{align}
    \nu_{k} &= -\ln\brackets{1-\frac{\sigma_{k}^{2}}{\norm{A}_{F}^{2}}}
        \nonumber\\
    \lambda_{k} &= \ln\brackets{1+\frac{\sigma_{k}^{2}}{\norm{A}_{F}^{2}}},
\end{align}
%
then
%
\begin{equation}
    \P\brackets{\overline{X}_{d}\ge \norm{A}_{F}^{2}\mu}
        \le \exp\brackets{-\frac{d\mu}{2}\parens{
            \mu-\braces{\nu_{1}+\cdots+\nu_{r}}}}
\end{equation}
%
decays exponentially in $d$ when
%
\begin{equation}
    \mu > 1 + \frac{\norm{A}_{2}^{2}}{\norm{A}_{F}^{2} - \norm{A}_{2}^{2}}.
\end{equation}
%
Similarly,
\begin{equation}
    \P\brackets{\overline{X}_{d}\le \norm{A}_{F}^{2}\mu}
        \le \exp\brackets{-\frac{d}{2}\parens{
            \braces{\lambda_{1}+\cdots+\lambda_{r}}-\mu}}
\end{equation}
%
decays exponentially in $d$ when $\mu\in[0,\ln2)$.
\end{thm}

\begin{proof}
When $r=1$ (that is, when $A$ is a rank-one matrix),
we can use Thm.~\ref{thm:Chernoff} to compute exponential
bounds on tail probabilities.
Because these bounds can be computed exactly, we assume $r\ge2$.

Clearly $X$ is a linear combination of chi-squared distributions,
so by properties of the moment generating function we have
%
\begin{equation}
    M_{\overline{X}_{d}}(t) =
    \prod_{k=1}^{r}
    \parens{1 - \frac{2\sigma_{k}^{2}}{d}t}^{-\frac{d}{2}}.
\end{equation}
%
Attempting to compute
%
\begin{equation}
    \min_{t>0} e^{-at}M_{\overline{X}_{d}}(t)
\end{equation}
%
will require factoring the roots of a degree $r$ polynomial in $t$;
a standard result of Galois theory is that this is not possible
in general for $r\ge5$.
Instead, we set
%
\begin{equation}
    \bar{t} = \frac{d}{2\norm{A}_{F}^{2}}.
\end{equation}
%
Then, if we set $a = \mu\norm{A}_{F}^{2}$ for $\mu>1$, we have
%
\begin{align}
    \P\brackets{\overline{X}_{d}\ge \mu\norm{A}_{F}^{2}}
    &\le \min_{t>0} \exp\parens{-\mu\norm{A}_{F}^{2}t} M_{\overline{X}_{d}}(t)
        \nonumber\\
    &\le \exp\parens{-\mu\norm{A}_{F}^{2}\bar{t}} M_{\overline{X}_{d}}(\bar{t})
        \nonumber\\
    &= \exp\parens{-\frac{\mu d}{2}}\prod_{k=1}^{r}
        \brackets{1-\frac{\sigma_{k}^{2}}{\norm{A}_{F}^{2}}}^{-\frac{d}{2}}
        \nonumber\\
    &= \exp\parens{-\frac{\mu d}{2}}\norm{A}_{F}^{rd} \prod_{k=1}^{r}
        \parens{A_{k}'}^{-d}.
\end{align}
%
Here, $A_{k}'$ is as defined in Eq.~\eqref{eq:rand_As_def}.
Similarly, for $\mu\in[0,1)$ and $a = \mu\norm{A}_{F}^{2}$, we have
%
\begin{align}
    \P\brackets{\overline{X}_{d}\le \mu\norm{A}_{F}^{2}}
    &\le \min_{t>0} \exp\parens{\mu\norm{A}_{F}^{2}t} M_{\overline{X}_{d}}(-t)
        \nonumber\\
    &\le \exp\parens{\mu\norm{A}_{F}^{2}\bar{t}} M_{\overline{X}_{d}}(-\bar{t})
        \nonumber\\
    &= \exp\parens{\frac{\mu d}{2}}\prod_{k=1}^{r}
        \brackets{1+\frac{\sigma_{k}^{2}}{\norm{A}_{F}^{2}}}^{-\frac{d}{2}}
        \nonumber\\
    &= \exp\parens{\frac{\mu d}{2}}\norm{A}_{F}^{rd} \prod_{k=1}^{r}
        \parens{A_{k}''}^{-d}.
\end{align}
%
$A_{k}''$ is defined in Eq.~\eqref{eq:rand_As_def}.
This proves the desired bounds as stated in the theorem.
Stronger tail probability bounds could be determined but we will
not pursue the matter here, for to do so may require
knowledge of the singular value decay.



We now focus on determining when the tail probabilities decay
exponentially in $d$.
Looking at the upper tail probability, we have
%
\begin{align}
    \P\brackets{\overline{X}_{d}\ge \mu\norm{A}_{F}^{2}}
    &\le \exp\parens{-\frac{\mu d}{2}}\norm{A}_{F}^{rd} \prod_{k=1}^{r}
        \parens{A_{k}'}^{-d} \nonumber\\
    &= \exp\parens{-\frac{d}{2}\brackets{\mu-\braces{\nu_{1}+\cdots+\nu_{r}}}},
\end{align}
%
where
%
\begin{equation}
    \nu_{k} = -\ln\brackets{1-\frac{\sigma_{k}^{2}}{\norm{A}_{F}^{2}}}.
\end{equation}
%
We will have exponential tail probability decay in $d$ if
%
\begin{equation}
    \nu_{1} + \cdots + \nu_{r} < \mu.
\end{equation}
%
We know $-\ln x$ is a convex function, so $-\ln(1-x)$ is convex on $[0,1)$.
For any $\alpha\in(0,1)$, we have
%
\begin{equation}
    \ln\parens{\frac{1}{1-x}}\le\frac{x}{\alpha}\ln\parens{\frac{1}{1-\alpha}}.
\end{equation}
%
Because $r\ge2$, we have $\sigma_{1} = \norm{A}_{2} < \norm{A}_{F}$ and
set $\alpha = \frac{\norm{A}_{2}^{2}}{\norm{A}_{F}^{2}} < 1$.
It now follows that
%
\begin{align}
    \nu_{1} + \cdots + \nu_{r}
        &\le \frac{1}{\alpha}\ln\parens{\frac{1}{1-\alpha}}
            \brackets{\frac{\sigma_{1}^{2}}{\norm{A}_{F}^{2}} + \cdots +
            \frac{\sigma_{r}^{2}}{\norm{A}_{F}^{2}}} \nonumber\\
    &= \frac{1}{\alpha}\ln\parens{1 + \frac{\alpha}{1-\alpha}} \nonumber\\
    &\le 1 + \frac{\alpha}{1-\alpha} \nonumber\\
    &= 1 + \frac{\norm{A}_{2}^{2}}{\norm{A}_{F}^{2} - \norm{A}_{2}^{2}},
\end{align}
%
where we used $\ln(1+x)\le x$ in the last inequality.
So long as
%
\begin{equation}
    \mu > 1 + \frac{\norm{A}_{2}^{2}}{\norm{A}_{F}^{2} - \norm{A}_{2}^{2}},
\end{equation}
%
we have exponential decay in $d$.

We now look at the lower tail probability.
For $\mu\in[0,1)$, we have
%
\begin{align}
    \P\brackets{\overline{X}_{d}\le \norm{A}_{F}^{2}\mu}
        &\le \exp\parens{\frac{d\mu}{2}} \norm{A}_{F}^{dr}
            \prod_{k=1}^{r}\parens{A_{k}''}^{-d} \nonumber\\
        &= \exp\brackets{\frac{d}{2}\braces{\mu - \parens{\lambda_{1}
            + \cdots + \lambda_{r}}}},
\end{align}
%
where
%
\begin{equation}
    \lambda_{k} = \ln\brackets{1 + \frac{\sigma_{k}^{2}}{\norm{A}_{F}^{2}}}.
\end{equation}
%
To have exponential decay in probability, we require
%
\begin{equation}
    \lambda_{1} + \cdots + \lambda_{r} > \mu \, .
\end{equation}
%
Now, we know
%
\begin{equation}
    \ln\parens{1 + x} \ge x\ln 2 \quad x\in\brackets{0,1},
\end{equation}
%
which implies
%
\begin{align}
    \lambda_{1} + \cdots + \lambda_{r}
        &\ge \frac{\sigma_{1}^{2}}{\norm{A}_{F}^{2}}\ln2
            + \cdots + \frac{\sigma_{r}^{2}}{\norm{A}_{F}^{2}}\ln2 \nonumber\\
        &= \ln 2.
\end{align}
%
Therefore, so long as $\mu<\ln2$, we have exponentially decaying tail
probabilities in $d$.

\end{proof}



We now analyze the situation for power iteration.
Let $\alpha\ge\frac{1}{2}$ and define
%
\begin{equation}
    Y_{\alpha} = \sigma_{1}^{2\alpha}\xi_{1}^{2} + \cdots +
        \sigma_{r}^{2\alpha}\xi_{r}^{2}.
\end{equation}
%
Additionally, let
%
\begin{equation}
    \overline{Y}_{d,\alpha} = \frac{1}{d}\brackets{Y_{1,\alpha} + \cdot
        + Y_{d,\alpha}},
    \label{eq:rand_averaged_power_RV}
\end{equation}
%
where $Y_{i,\alpha}$ are independent and identically distributed realizations
of $Y_{\alpha}$.
Naturally, $\E(Y_{\alpha}) = \E(\overline{Y}_{d,\alpha}) =
\norm{A}_{s,2\alpha}^{2\alpha}$
and it clear we have
%
\begin{align}
    \V(\overline{Y}_{d,\alpha}) &= \frac{1}{d}\V(Y_{\alpha}) \nonumber\\
        &= \frac{2\norm{A}_{s,4\alpha}^{4\alpha}}{d}.
\end{align}
%
We have the analogous theorem:

\begin{thm}[Probabilistic Error Bounds for Power Iteration]
\label{thm:prob_err_bound_power}
Given $\overline{Y}_{d}$ as defined in Eq.~\eqref{eq:rand_averaged_power_RV}
with $r\ge2$ and
%
\begin{align}
    \bar{\nu}_{k} &= -\ln\brackets{1-\frac{\sigma_{k}^{2\alpha}}{
        \norm{A}_{s,2\alpha}^{2\alpha}}}
        \nonumber\\
    \bar{\lambda}_{k} &= \ln\brackets{1+\frac{\sigma_{k}^{2\alpha}}{
        \norm{A}_{s,2\alpha}^{2\alpha}}},
\end{align}
%
then
%
\begin{equation}
    \P\brackets{\overline{Y}_{d}\ge \norm{A}_{s,2\alpha}^{2\alpha}\mu}
        \le \exp\brackets{-\frac{d\mu}{2}\parens{
            \mu-\braces{\bar{\nu}_{1}+\cdots+\bar{\nu}_{r}}}}
\end{equation}
%
decays exponentially in $d$ when
%
\begin{equation}
    \mu > 1 + \frac{\norm{A}_{s,\infty}^{2\alpha}}{
        \norm{A}_{s,2\alpha}^{2\alpha} - \norm{A}_{2,\infty}^{2\alpha}}.
\end{equation}
%
Similarly,
\begin{equation}
    \P\brackets{\overline{Y}_{d}\le \norm{A}_{s,2\alpha}^{2\alpha}\mu}
        \le \exp\brackets{-\frac{d}{2}\parens{
            \braces{\bar{\lambda}_{1}+\cdots+\bar{\lambda}_{r}}-\mu}}
\end{equation}
%
decays exponentially in $d$ when $\mu\in[0,\ln2)$.
\end{thm}

\begin{proof}
The proof follows the same argument as that of 
Thm.~\ref{thm:prob_err_bound}.
We obtain the results of Thm.~\ref{thm:prob_err_bound}
when $\alpha=1$, as expected.
\end{proof}

The above theorem ensures that independent realizations
of Eq.~\eqref{eq:rand_new_power_error_bound} closely
match the expected value.
This will make it possible for us to bound $\norm{\cdot}_{s,p}$
depending on the power as well as allowing for a relative
stopping criterion in these norms.




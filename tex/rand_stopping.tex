\section{Stopping Criteria}
\label{sec:rand_stop_crit}

The unitary invariance of the 2-norm combined with it being
the definition of the operator (matrix) norm makes it
desirable to bound $\norm{(I-QQ^{*})A}_{2}$.
The main challenge of the 2-norm stems from the computational
complexity required to compute it exactly.
Because of this, the F-norm is the next norm that we may wish
to bound, and it is easy to compute if we have explicit access to the matrix
in question; furthermore, it trivially bounds the 2-norm.
If $A$ is large or implicitly defined, $\norm{A}_{F}$ may still be expensive
to compute.
Additionally, the exact norm is frequently not required so much as
an accurate approximation of it.
We will be making comparisons with algorithms
in~\cite{martinsson2016randomized,yu2018efficient}, so we reproduce
their blocked versions here.
Alg.~\ref{alg:rand_low_rank_MV}
is from~\cite[Figure 2]{martinsson2016randomized} (the MV Algorithm,
named after the authors)
and Alg.~\ref{alg:rand_low_rank_YGL}
is from~\cite[Algorithm 2]{yu2018efficient} (the YGL Algorithm,
named after the authors).
Algs.~\ref{alg:rand_low_rank_MV} and \ref{alg:rand_low_rank_YGL} originally
used $Q = \texttt{orth}(A)$ to compute the orthonormal basis
of a matrix $A$; this work uses the function \texttt{qr} in our notation
because it is an unpivoted QR factorization,
and our new stopping criterion will require both $Q$ and $R$ factors
of the unpivoted QR factorization.
A variation of iterated Gram-Schmidt orthogonalization can be found in
Line~\ref{alg_line:rand_MV_GS}  of Alg.~\ref{alg:rand_low_rank_MV}  and
Line~\ref{alg_line:rand_YGL_GS} of Alg.~\ref{alg:rand_low_rank_YGL}
to help $Q$ retaining numerical orthogonality.
It is explicitly stated in~\cite[Remark 5]{martinsson2016randomized}
that the fact $R$ is not used in the \texttt{qr} routine
(original \texttt{orth}).
From our work here, we see the $R$ factor helps ensure we do
not take too many random samples by noting when the $R$-values become small.
Even so, no pivoting is required.


\begin{algorithm}[p]
\caption{Randomized Block Low-Rank Approximation (MV)
    \cite[Figure 2]{martinsson2016randomized}}
\label{alg:rand_low_rank_MV}
\begin{algorithmic}[1]
\Function{randQB\_b\_MV}{$A$,$\eps$,$d$}
    \For {$k=1,2,3,\cdots$}
        \State $\Omega_{k} = \texttt{rand}(n,d)$
        \State $Q_{k} = \texttt{qr}(A\Omega_{k})$
        \State $Q_{k} = \texttt{qr}(Q_{k}-\sum_{j=1}^{k-1}Q_{j}Q_{j}^{*}Q_{k})$
            \label{alg_line:rand_MV_GS}
        \State $B_{k} = Q_{k}^{*}A$
        \State $A = A - Q_{k}B_{k}$
        \If {$\norm{A} < \eps$}
            \State \textbf{stop}
        \EndIf
    \EndFor
\State $Q = \begin{bmatrix} Q_{1} & \cdots & Q_{k} \end{bmatrix}$
\State $B = \begin{bmatrix} B_{1}^{*} & \cdots & B_{k}^{*} \end{bmatrix}^{*}$
\State \Return $Q$, $B$
\EndFunction
\end{algorithmic}
\end{algorithm}

\begin{algorithm}[p]
\caption{Randomized Block Low-Rank Approximation (YGL)
    \cite[Algorithm 2]{yu2018efficient}}
\label{alg:rand_low_rank_YGL}
\begin{algorithmic}[1]
\Function{randQB\_b\_YGL}{$A$,$\eps$,$d$}
\State $Q = [\,]$; $B = [\,]$
\State $E = \norm{A}_{F}^{2}$
    \For {$k=1,2,3,\cdots$}
        \State $\Omega_{k} = \texttt{rand}(n,d)$
        \State $Q_{k} = \texttt{qr}(A\Omega_{k} - Q(B\Omega_{k}))$
        \State $Q_{k} = \texttt{qr}(Q_{k} -Q (Q^{*}Q_{k}))$
            \label{alg_line:rand_YGL_GS}
        \State $B_{k} = Q_{k}^{*}A$
        \State $Q = \begin{bmatrix} Q & Q_{k} \end{bmatrix}$
        \State $B = \begin{bmatrix} B^{*} & B_{k}^{*} \end{bmatrix}^{*}$
        \State $E = E - \norm{B_{k}}_{F}^{2}$
        \If {$E < \eps^{2}$}
            \State \textbf{stop}
        \EndIf
    \EndFor
\State \Return $Q$, $B$
\EndFunction
\end{algorithmic}
\end{algorithm}


One stopping criterion is based on the following lemma~\cite{RandomReview2011}:

\begin{lem}[HMT Error Bound;
    Lemma 4.1 in~\cite{RandomReview2011}]
\label{thm:rand_hmt_bound}
Let $B\in\R^{m\times n}$. Fix a positive integer $d$ and $\alpha>1$.
Draw an independent family of standard Gaussian vectors
$\braces{\omega_{k}}_{k=1}^{d}$.
Then
%
\begin{equation}
    \norm{B}_{2} \le \alpha\sqrt{\frac{2}{\pi}}\max_{k=1,\cdots,d}
        \norm{B\omega_{k}}_{2}
    \label{eq:rand_hmt_bound}
\end{equation}
%
with probability $1-\alpha^{-d}$.
\end{lem}

\noindent
The reference to ``HMT'' comes from the first letters of
the authors' last names in~\cite{RandomReview2011}
and is used in~\cite[Alg.~4.2]{RandomReview2011}
(Adaptive Randomized Range Finder).
Here, $\alpha$ can be viewed as a trade-off parameter: larger values
ensure a smaller probability of failure.
It turns out, though, this overestimation is significant; this will
be shown for a number of matrices in Sec.~\ref{sec:rand_exp}.
We can find no references to randomized lower bounds of $\norm{B}_{2}$;
this would be helpful in producing an order-of-magnitude estimate
or estimates relating to the variance or other moments of the random variable.
This method is used in~\cite{liu2016parallel,xi2014fast,saibaba2016randomized}
as a stopping criterion for adaptive randomized algorithms
for structured matrices
(the same situation where this work was originally developed).

The authors in \cite{martinsson2016randomized} note the drawback of
Lem.~\ref{thm:rand_hmt_bound} of this over estimation and instead
use the explicit bound $\norm{A}<\eps$ in Alg.~\ref{alg:rand_low_rank_MV},
where we note $A$ is updated every time after having known
information subtracted out during each iteration.
The authors said that $\norm{A}_{2}$ could be used but noted
that $\norm{A}_{F}$ may be preferred for its easy computation.
One drawback (noted in~\cite{yu2018efficient}) is that this requires 
access to a dense copy of $A$.
This is problematic with regards to memory because we may
want to keep an unmodified version of the original matrix; furthermore,
if $A$ is originally sparse, it will immediately become dense
after subtracting the first approximation,
greatly increasing memory requirements.

This resulted in the development of the stopping criterion
in~\cite{yu2018efficient}.
Instead of storing a dense matrix with the unused information
of $A$ to keep track of the error, it is possible to compute
the error just based on the difference in F-norm.
This is made explicit in the next theorem, which we call
the ``YGL Error Bound'' because of the first letters
of the authors' last names.
Calling this an error bound is slightly misleading because the
error is exact, not bounded, but a better title escapes us at this time.

\begin{thm}[YGL Error Bound;
    Theorem 1.1 in~\cite{yu2018efficient}]
\label{thm:rand_ygl_bound}
Let $A\in\R^{m\times n}$ and $Q\in\R^{n\times k}$ be orthogonal.
If $B = Q^{*}A$ then
%
\begin{equation}
    \norm{A - QB}_{F}^{2} = \norm{A}_{F}^{2} - \norm{B}_{F}^{2}.
    \label{eq:rand_ygl_bound}
\end{equation}
\end{thm}

\noindent
If one builds up $Q$ in blocks, then we can compute the true F-norm error
provided $\norm{A}_{F}$ is already known.
Because this value only needs to be computed once, it may not 
be too expensive;
however, this only works if the matrix is explicitly, not implicitly, defined.
Additionally, Thm.~\ref{thm:rand_ygl_bound} is only useful
for relative tolerances down to $O(\sqrt{\epsm})$,
where $\epsm$ is machine precision,
as noted in~\cite[Theorem 3]{yu2018efficient}.
It is possible, using techniques such as compensated
summation~\cite[Chapter 4]{HighamASNA}, that these limitations may be
lifted, but this would be difficult in the parallel setting.
This may be the first explicit reference to a relative error bound
in the literature, though we seek to remove the relative tolerance restrictions.



\section{Previous Probabilistic Bounds}

We now present results from~\cite{gu2015subspace}, although
looser bounds from~\cite{RandomReview2011} may be more well-known.
The important portions of the bounds will be emphasized here;
the original paper should be referenced for the full results and proofs.

\begin{thm}[Average 2-Norm and F-Norm Error;
    Theorem 5.7 in~\cite{gu2015subspace}]
\label{thm:rand_ave_2_F_norm_err_gu}
Let $\Omega$ be a Gaussian random matrix with $k+p$ columns for
$p\ge2$, $S = (AA^{*})^{q}A\Omega$, and $Q = \texttt{qr}(S)$.
Then
%
\begin{align}
    \E\norm{A-QQ^{*}A}_{2} &\le
        \sqrt{\sigma_{k+1}^{2} + C(A,k,p,q)} \nonumber\\
    \E\norm{A-QQ^{*}A}_{F} &\le
        \sqrt{\parens{\sum_{j=k+1}^{n}\sigma_{j}^{2}} + D(A,k,p,q)}
\end{align}
%
Here, $C$ and $D$ are constants which decay exponentially as $q$ increases.
\end{thm}

\noindent
The above theorem states the expected value of the error is close to optimal.
Similar results hold for upper tail bounds:

\begin{thm}[Probability Tail Bounds in 2-Norm and F-Norm;
    Theorem 5.8 in~\cite{gu2015subspace}]
Let the assumptions of Thm.~\ref{thm:rand_ave_2_F_norm_err_gu} hold.
Then if $0<\Delta\ll1$,
%
\begin{align}
    \norm{A-QQ^{*}A}_{2} &\le
        \sqrt{\sigma_{k+1}^{2} + C(A,k,p,q,\Delta)} \nonumber\\
    \norm{A-QQ^{*}A}_{F} &\le
        \sqrt{\parens{\sum_{j=k+1}^{n}\sigma_{j}^{2}} + D(A,k,p,q,\Delta)}
\end{align}
%
hold for probability $1-\Delta$.
$C$ and $D$ are constants which decay exponentially as $q$ increases.
\end{thm}

\noindent
Additionally, \cite[Theorem 5.6]{gu2015subspace} bounds deviations
of the singular values of $QQ^{*}A$ from the singular values of $A$.

These are useful results and help explain why randomized methods
perform better than expected if one looks at similar
bounds in~\cite{RandomReview2011}.
The downside is that in practice they are not helpful because
we must know the singular values.

In these theorems, $q$ refers to the possibility of using
power iteration in order to increase the quality of the approximation;
this is a well-known technique in randomized numerical linear
algebra~\cite{RandomReview2011,yu2018efficient,
martinsson2016randomized} and helps ensure the larger singular values
and corresponding singular vectors are matched more
closely~\cite{gu2015subspace,saibaba2019randomized}.
Power iteration is not mentioned in~\cite{randomHSSLBL}
because it is not clear how to use it in
randomized HSS construction without greatly increasing the
overall computational cost; it is likely this also holds in general
for the randomized construction of structured matrices
because we do not have explicit access to matrix subblocks.
This lead to the desire for an accurate method to determine
low-rank approximations which could be used with small relative
tolerances (for example, tolerances as small as
$10^{-5}$ in single precision and $10^{-14}$ in double precision).

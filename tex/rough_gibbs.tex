\section{Gibbs Phenomenon and Smooth Cutoff Filters}

We now turn our attention from interpolating smooth functions
to interpolating rough functions.

The Gibbs phenomenon~\cite[Chapter 2]{zygmund} is a well-known problem
when one attempts to approximate a discontinuous function by a finite sum of
continuous functions, frequently chosen to be a Fourier series;
this is also true for Chebyshev expansions.
Recent work has been focused on determining filters
which smoothly cutoff the Fourier series; one review article
is~\cite{gottlieb1997gibbs}.
The filters $\sigma$ in the review article take the form
%
\begin{equation}
    f_{N}^{\sigma}(x) =
        \sum_{k=-\infty}^{\infty}\hat{f}_{k}\sigma\parens{\frac{k}{N}}e^{ikx},
\end{equation}
%
where $\sigma$ smoothly cuts off the higher order frequencies;
a more precise definition is given below.
This has a similar form of the smooth cutoff approximations
for the MSN method found in~\cite{msnInterp,msnBirkhoff}
as well as our proofs of Fast MSN convergence in
Chapter~\ref{chap:cvip_converge}.
In this way, MSN interpolation could be thought of as a smoothing
filter which acts in such a way so as to retain equality
at certain nodes (in this case, the Chebyshev interpolation nodes).
We will choose $s\in\braces{1,2,3,4,5,6}$ for our simulations,
although in practice $s$ can be any real number.

From~\cite{gottlieb1997gibbs}, we say a filter $\sigma$ is of order $p$ if
%
\begin{itemize}
\item $\sigma$ is an even function,
\item $\sigma(0) = 1$, $\sigma^{(\ell)}(0) = 0$ for $1\le\ell\le p-1$,
\item $\sigma(\eta) = 0$ for $\abs{\eta}\ge1$, and
\item $\sigma\in C^{p-1}$ for $\eta\in\parens{-\infty,\infty}$.
\end{itemize}

\noindent
We reproduce some of the filters from~\cite{gottlieb1997gibbs}
which we will use to compare with our MSN results.
In all instances, we set $\sigma(\eta) = 0$ when $\eta>1$.

\begin{itemize}
\item Chebyshev interpolation:
%
\begin{equation}
    \sigma_{0}(\eta) = 1.
\end{equation}
%
This results in no filtering but is kept as a baseline comparison.
Clearly $\sigma_{0}$ is not continuous at $\eta=1$.

\item The Fej\'{e}r filter:
%
\begin{equation}
    \sigma_{1}(\eta) = 1 - \eta.
\end{equation}

\item The Lanczos filter:
%
\begin{equation}
    \sigma_{2}(\eta) = \frac{\sin(\pi\eta)}{\pi\eta}.
\end{equation}
%
This is the normalized sinc function.

\item The Raised Cosine Filter:
%
\begin{equation}
    \sigma_{3}(\eta) = \frac{1}{2}\parens{1 + \cos(\pi\eta)}.
\end{equation}

\item The Sharpened Raised Cosine Filter:
%
\begin{equation}
    \sigma_{4}(\eta) = \sigma_{3}^{4}(\eta)\parens{
        35 - 84\sigma_{3}(\eta) + 70\sigma_{3}^{2}(\eta)
        - 20\sigma_{3}^{3}(\eta)}.
\end{equation}
%
This is an order 8 filter and will be denoted as ``Cos8'' in error plots.

\item The Exponential Filter of order $p$:
%
\begin{equation}
    \sigma_{5}(\eta) = \exp\parens{-\alpha\eta^{p}}.
\end{equation}
%
We will always take $p$ to be even and $\alpha = -\ln(\eps_{\text{mach}})$,
so that the largest frequency will be $O(\eps_{\text{mach}})$.
Naturally, $\sigma_{5}$ is not actually continuous at $\eta=1$.
\end{itemize}

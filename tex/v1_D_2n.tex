\section{1D \CV{} Derivative Matrix: $2n+1$ Columns; First Factorization}
\label{sec:CV_D_1D_2n_F2}

We now look at computing a factorization for $V' = VD$.
From our work in Sec.~\ref{sec:CV_diff}, we know that $UWD$ is highly
structured and similar to $W$ in that it only has one ``diagonal''.
If we include the diagonal scaling $D_{s}^{-1}$, we see
%
\begin{equation}
    UWDD_{s}^{-1} =
    \begin{bmatrix}
    0 & \frac{1}{2^{s}} & & & & & & & \frac{2n-1}{\parens{2n}^{s}} & 0 \\
      & & \ddots & & & & \iddots & \\
      & & & \frac{n-1}{n^{s}} & & \frac{n+1}{\parens{n+2}^{s}} \\
      & & & & \frac{n}{\parens{n+1}^{s}} \\
    \end{bmatrix},
\end{equation}
%
We now want zero out the right half of the $UWDD_{s}^{-1}$.
We let
%
\begin{align}
    \tau_{k} &= -\parens{\frac{2n+1-k}{k-1}}\parens{\frac{k}{2n+2-k}}^{s}
        \quad k\in\braces{2,\cdots,n} \nonumber\\
    c_{k} &= \frac{1}{\sqrt{1 + \tau_{k}^{2}}} \nonumber\\
    s_{k} &= \tau_{k}c_{k}.
    \label{eq:v1_D_2n_Givens_cs_def}
\end{align}
%
Using these $c$ and $s$ values, our full set of Givens rotations is
%
\begin{equation}
    G = \begin{bmatrix}
        1     &       &        &       &      &       &        &       &      \\
              & c_{2} &        &       &      &       &        & s_{2} &      \\
              &       & \ddots &       &      &       & \iddots&       &      \\
              &       &        & c_{n} &      & s_{n} &        &       &      \\
              &       &        &       &    1 &       &        &       &      \\
              &       &        & -s_{n}&      & c_{n} &        &       &      \\
              &       & \iddots&       &      &       & \ddots &       &      \\
              &-s_{2} &        &       &      &       &        & c_{2} &      \\
              &       &        &       &      &       &        &       & 1    \\
        \end{bmatrix},
    \label{eq:vand_D_Givens}
\end{equation}
%
and we set
%
\begin{align}
    \gamma_{k} &= \parens{k-1}k^{-s}\sqrt{1 + \parens{\frac{2n+1-k}{k-1}}^{2}
        \parens{\frac{k}{2n+2-k}}^{2s}} \quad k\in\braces{2,\cdots,n}
            \nonumber\\
    \gamma_{n+1} &= n\parens{n+1}^{-s}.
    \label{eq:v1_I_2n_Givens_gamma_def}
\end{align}
%
Using all of the above information, we see
%
\begin{align}
    UWDD_{s}^{-1}G &= \begin{bmatrix} 0 & \Gamma & 0 \end{bmatrix} \nonumber\\
    \Gamma &= \diag\parens{\gamma_{2},\gamma_{3}\cdots,\gamma_{n+1}}.
\end{align}
%
If $\Pi$ is the circular downshift permutation matrix, then
%
\begin{equation}
    UWDD_{s}^{-1}G\Pi = \begin{bmatrix} \Gamma & 0 \end{bmatrix}
\end{equation}
%
is the desired LQ factorization.



\section{1D \CV{} Derivative Matrix: $2n+1$ Columns; Second Factorization}
\label{sec:CV_D_1D_2n_F1}

In this section we will look at another method for factorizing $V'$.
This method was developed first but we think the ``First Factorization''
is more useful in practice;
this factorization is recorded here on the chance
it may be beneficial in the future.

From pervious work, we see
%
\begin{align}
    &WDD_{s}^{-1}\parens{1,:} = \nonumber\\
        &\quad\begin{bmatrix}
            0 & 1\cdot2^{-s} & 0 & 3\cdot4^{-s} & 0 &
            5\cdot6^{-s} & \cdots &
            \parens{2n-3}\parens{2n-2}^{-s} & 0 &
            \parens{2n-1}\parens{2n}^{-s} & 0
        \end{bmatrix} \nonumber\\
    &WDD_{s}^{-1}\parens{k,k+1\!:\!2n+1-k} = \nonumber\\
        &\quad\begin{bmatrix}
            2k\parens{k+1}^{-s} & 0 &
            2\parens{k+2}\parens{k+3}^{-s} & 0 & \cdots & 0 & 
            2\parens{2n-k}\parens{2n+1-k}^{-s} \end{bmatrix}
            \nonumber\\
    &\quad k\ge2.
\end{align}
%
The nonzero entries are (essentially) constant along the columns;
the first row is off by a factor of $\frac{1}{2}$.
To fix this, we define
%
\begin{equation}
    \widehat{D} = \diag\parens{1,\frac{1}{2},\cdots,\frac{1}{2}}.
\end{equation}
%
Now, we look at the case when $n=5$ to clearly see the structure:
%
\begin{align}
    &\widehat{D}WDD_{s}^{-1} = \nonumber\\
        &\begin{bmatrix}
            0 & 1\cdot2^{-s} & & 3\cdot4^{-s} & &
            5\cdot6^{-s} & & 7\cdot8^{-s} & &
            9\cdot10^{-s} & 0 \\
            & & 2\cdot3^{-s} & & 4\cdot5^{-s} & & 6\cdot7^{-s} & &
            8\cdot9^{-s} & \\
            & & & 3\cdot4^{-s} & & 5\cdot6^{-s} & & 7\cdot8^{-s} & 
            & & & \\
            & & & & 4\cdot5^{-s} & & 6\cdot7^{-s} & & & & & \\
            & & & & & 5\cdot6^{-s} & & & & & & \\
        \end{bmatrix}.
\end{align}


We begin by applying the same set of Givens rotations $G$
from Eq.~\eqref{eq:vand_D_Givens}.
When $n$ is even, then
%
\begin{equation}
    \widehat{D}WDD_{s}^{-1}G = \begin{bmatrix}
        0 & \gamma_{2} & 0 & \gamma_{4} & 0 & \gamma_{6} &
            \cdots & \gamma_{n-2} & 0 & \gamma_{n} &
            0 & 0 & \cdots & 0 \\
        & & \gamma_{3} & 0 & \gamma_{5} & 0 &
            \cdots & 0 & \gamma_{n-1} & 0 & \gamma_{n+1} & 0 & \cdots & 0 \\
        & & & \gamma_{4} & 0 & \gamma_{6} &
            \cdots & \gamma_{n-2} & 0 & \gamma_{n} & 0 & 0 & \cdots & 0 \\
        & & & & \gamma_{5} & 0 & 
            \cdots & 0 & \gamma_{n-1} & 0 & \gamma_{n+1} & 0 & \cdots & 0 \\
        & & & & & \gamma_{6} &
            \cdots & \gamma_{n-2} & 0 & \gamma_{n} & 0 & 0 & \cdots & 0 \\
        & & & & & & \ddots & & \vdots & & \vdots \\
        & & & & & & & \gamma_{n-2} & 0 & \gamma_{n} & 0 & 0 & \cdots & 0 \\
        & & & & & & & & \gamma_{n-1} & 0 & \gamma_{n+1} & 0 & \cdots & 0 \\
        & & & & & & & & & \gamma_{n} & 0 & 0 & \cdots & 0 \\
        & & & & & & & & & & \gamma_{n+1} & 0 & \cdots & 0 \\
        \end{bmatrix}.
\end{equation}
%
If $n$ is odd, then
%
\begin{equation}
    \widehat{D}WDD_{s}^{-1}G = \begin{bmatrix}
        0 & \gamma_{2} & 0 & \gamma_{4} & 0 & \gamma_{6} &
            \cdots & 0 & \gamma_{n-1} & 0 & \gamma_{n+1} &
            0 & \cdots & 0 \\
        & & \gamma_{3} & 0 & \gamma_{5} & 0 &
            \cdots & \gamma_{n-2} & 0 & \gamma_{n} & 0 & 0 & \cdots & 0 \\
        & & & \gamma_{4} & 0 & \gamma_{6} &
            \cdots & 0 & \gamma_{n-1} & 0 & \gamma_{n+1} & 0 & \cdots & 0 \\
        & & & & \gamma_{5} & 0 & 
            \cdots & \gamma_{n-2} & 0 & \gamma_{n} & 0 & 0 & \cdots & 0 \\
        & & & & & \gamma_{6} &
            \cdots & 0 & \gamma_{n-1} & 0 & \gamma_{n+1} & 0 & \cdots & 0 \\
        & & & & & & \ddots & & \vdots & & \vdots \\
        & & & & & & & \gamma_{n-2} & 0 & \gamma_{n} & 0 & 0 & \cdots & 0 \\
        & & & & & & & & \gamma_{n-1} & 0 & \gamma_{n+1} & 0 & \cdots & 0 \\
        & & & & & & & & & \gamma_{n} & 0 & 0 & \cdots & 0 \\
        & & & & & & & & & & \gamma_{n+1} & 0 & \cdots & 0 \\
        \end{bmatrix}.
\end{equation}

In both cases, we see that the matrix is upper triangular,
and we wish to put it in lower triangular form.
Because of the similarities between the two cases ($n$ even or odd),
we will focus on the case when $n$ is even. This is solely
a choice of convenience and will not affect the results.

To convert this into a lower triangular matrix, we see that nonzero
even and odd column pairs have nonzero terms following the form
%
\begin{equation}
    \begin{bmatrix}
        \gamma & \del \\
        \gamma & \del \\
        \vdots & \vdots \\
        \gamma & \del \\
        0 & \del \\
    \end{bmatrix}.
\end{equation}
%
In both cases we can use Givens rotations to zero out of the second
column. We will repeatedly do this and work backward to obtain the
LQ factorization.
After a permutation of the columns,
the matrix will be in lower triangular form.

We define
%
\begin{align}
    \del_{n+1} &= \gamma_{n+1} \nonumber\\
    \del_{n} &= \gamma_{n} \nonumber\\
    \del_{k} &= \gamma_{k}\sqrt{1 + \parens{\frac{\del_{k+2}}{\gamma_{k}}}^{2}}
        \quad k\in\braces{2,\cdots,n-1} \nonumber\\
    \xi_{k} &= -\frac{\del_{k+2}}{\gamma_{k}} \quad k\in\braces{2,\cdots,n-1}
        \nonumber\\
    \mu_{k} &= \frac{1}{\sqrt{1 + \xi_{k}^{2}}}
        \nonumber\\
    \nu_{k} &= \xi_{k}\mu_{k}.
\end{align}

\noindent
Letting
%
\begin{align}
    H_{k} &=
        \begin{bmatrix}
            I_{k-1} \\
            & \mu_{k} & & \nu_{k} \\
            & & 1 & \\
            & -\nu_{k} & & \mu_{k} \\
            & & & & I_{2n-1-k} \\
        \end{bmatrix} \nonumber\\
    H &= H_{n-1}H_{n-2}\cdots H_{3}H_{2},
\end{align}
%
we see
%
\begin{equation}
    \widehat{D}WDD_{s}^{-1}GH\Pi = \begin{bmatrix} L & 0 \end{bmatrix},
\end{equation}
%
where $\Pi$ is the circular downshift permutation matrix as before, $0$ is
an $n\times n+1$ matrix of zeros, and $L$ is a lower triangular matrix
with
%
\begin{align}
    L(2k+\ell,\ell) &= \parens{-1}^{k}\del_{2k+\ell}\nu_{2k+\ell-2}
        \nu_{2k+\ell-4}\cdots\nu_{\ell} \quad \ell\in\braces{2,3},k\ge0,
        2k+\ell\le n
        \nonumber\\
    L(2k+\ell,\ell) &= \parens{-1}^{k}\del_{2k+\ell}\nu_{2k+\ell-2}
        \nu_{2k+\ell-4}\cdots\nu_{\ell}\mu_{\ell-2} \quad \ell\ge4,k\ge0,
        2k+\ell\le n.
\end{align}
%
As an example, we show the case when $n=8$:
%
\begin{equation}
    L = \begin{bmatrix}
        \del_{2} \\
        & \del_{3} \\
        -\del_{4}\nu_{2} & & \del_{4}\mu_{2} \\
        & -\del_{5}\nu_{3} & & \del_{5}\mu_{3} \\
        \del_{6}\nu_{4}\nu_{2} & & -\del_{6}\nu_{4}\mu_{2} & & \del_{6}\mu_{4}\\
        & \del_{7}\nu_{5}\nu_{3}& & -\del_{7}\nu_{5}\mu_{3} & &\del_{7}\mu_{5}\\
        -\del_{8}\nu_{6}\nu_{4}\nu_{2} & & \del_{8}\nu_{6}\nu_{4}\mu_{2} & &
        -\del_{8}\nu_{6}\mu_{4} & & \del_{8}\mu_{6} \\
        & -\del_{9}\nu_{7}\nu_{5}\nu_{3} & & \del_{9}\nu_{7}\nu_{5}\mu_{3} & &
        -\del_{9}\nu_{7}\mu_{5} & & \del_{9}\mu_{7} \\
    \end{bmatrix}.
\end{equation}

\noindent
A direct computation will show us
%
\begin{align}
    \del_{k} &= ||\widehat{D}WDD_{s}^{-1}\parens{k-1,:}||_{2}
        \quad k\in\braces{2,\cdots,n+1},
\end{align}
%
but it is easier to see that the only nonzero entry of the $k$th row
of $\widehat{D}WDD_{s}^{-1}GH_{n}\cdots H_{k+1}$ is $\del_{k}$, and
the 2-norm is rotationally invariant.

Working through some calculations we find that $L^{-1}$ has a very simple
form: bidiagonal matrix with nonzero entries on the second sub-diagonal.
%
\begin{align}
    &L^{-1} = \nonumber\\
    &\begin{bmatrix}
        \del_{2}^{-1} & & & & & & & & \\
        0 & \del_{3}^{-1}& & & & & & & \\
        \del_{2}^{-1}\xi_{2} & 0 & \parens{\del_{4}\mu_{2}}^{-1} 
            & & & & & \\
        & \del_{3}^{-1}\xi_{3} & 0 & \parens{\del_{5}\mu_{3}}^{-1}
            & & & & & \\
        & & \ddots & & \ddots & & & & \\
        & & & & & \parens{\del_{n-1}\mu_{n-3}}^{-1} & & \\
        & & & & \ddots & 0 & 
            \parens{\del_{n}\mu_{n-2}}^{-1} & \\
        & & & & & \del_{n-1}^{-1}\xi_{n-1} & 0 &
            \parens{\del_{n+1}\mu_{n-1}}^{-1} \\
    \end{bmatrix}
\end{align}
%
Because of its simpler form, it will be easier to work with $L^{-1}$ directly
rather than $L$.
If we $P$ is the permutation matrix which permutes the even and odd
rows together in increasing order, then it is clear
%
\begin{equation}
    PL^{-1}P = \diag(L_{1},L_{2}),
\end{equation}
%
where $L_{1}$ and $L_{2}$ are lower triangular matrices
with nonzeros only on the sub-diagonal, implying they are
semiseparable~\cite[Chapter 12]{gvl4}.

The methods we applied here could also be applied when $V'$
has $3n+1$ or $4n+1$ columns, but we will not pursue those factorizations
 at this time.
Left multiplying by $U$ greatly simplifies matters because
then $UWD$ and $UWDD_{s}^{-1}$ have orthogonal rows;
this is our reasoning for preferring the other factorization.



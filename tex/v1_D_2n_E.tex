\section{1D \CV{} Derivative Matrix: $2n+1$ Columns with Point Interpolation;
    First Factorization}
\label{sec:CV_D_1D_2n_E_Point}

In this section we compute the resulting LQ factorization for the
interpolation problem
%
\begin{equation}
    \begin{cases}
        p'(z_{k}^{n}) = f'(z_{k}^{n}), &\quad k\in\braces{1,\cdots,n} \\
        p(0) = f(0)
    \end{cases}.
\end{equation}
%
Ensuring equality at $p(0) = f(0)$ gives the MSN approximation the
correct constant term.
If we denote the linear interpolation requirement by $A$, then
we have
%
\begin{equation}
    \begin{bmatrix}
        A \\ UWD \end{bmatrix} D_{s}^{-1}G
     = \begin{bmatrix}
        1 & 0 & 0 \\ 0 & \Gamma & 0 \end{bmatrix},
\end{equation}
%
where $G$ is from Eq.~\eqref{eq:vand_D_Givens}
and we remember $Ge_{1} = e_{1}$.
This is the LQ factorization.




\section{1D \CV{} Derivative Matrix: $2n+1$ Columns with Endpoint Interpolation;
    First Factorization}
\label{sec:CV_D_1D_2n_E_F2}

In this section we compute the solution when interpolating
derivatives on Chebyshev nodes and function values on the endpoints:
%
\begin{equation}
    \begin{cases}
        p'(z_{k}^{n}) = f'(z_{k}^{n}), &\quad k\in\braces{1,\cdots,n} \\
        p(z) = f(z), &\quad z\in\braces{-1,1}
    \end{cases}.
\end{equation}
%
We will use tools from Sec.~\ref{sec:CV_D_1D_2n_F2}.
When $n$ is even, we see
%
\begin{align}
    &\begin{bmatrix} UWD \\ \overline{A} \end{bmatrix}D_{s}^{-1}G\Pi
        = \nonumber\\
    &\begin{bmatrix}
            \gamma_{2} & \\
            & \gamma_{3} & \\
            & & & \\
            &   &   & \ddots \\
            & & & & \gamma_{n} & \\
            & & & & & \gamma_{n+1} & \\
            \alpha_{2} & & \alpha_{4} & \cdots & \alpha_{n} & &
                \alpha_{n+2} & & \cdots & \alpha_{2n} & & \\
            & \alpha_{3} & & \cdots & & \alpha_{n+1} &
                & \alpha_{n+3} & \cdots & & \alpha_{2n+1} & \alpha_{1} \\
            \end{bmatrix},
\end{align}
%
where $\overline{A} = C_{1}^{-1}A$ with definitions coming from 
Eqs.~\eqref{eq:vand_pm1} and \eqref{eq:vand_pm1_C}.
We must then compute 2 Householder reflectors by setting
%
\begin{align}
    v_{1}^{*} &= \begin{bmatrix} \alpha_{n+2} & 0 & \alpha_{n+4} & \cdots &
                \alpha_{2n} & 0 & 0 \end{bmatrix} \nonumber\\
    v_{2}^{*} &= \begin{bmatrix} 0 & \alpha_{n+3} & 0 & \cdots &
                0 & \alpha_{2n+1} & \alpha_{1} \end{bmatrix} \nonumber\\
    u_{1} &= \frac{1}{\norm{v_{1} - \norm{v_{1}}_{2}e_{1}}_{2}}
                \parens{v_{1} - \norm{v_{1}}_{2}e_{1}} \nonumber\\
    u_{2} &= \frac{1}{\norm{v_{2} - \norm{v_{2}}_{2}e_{2}}_{2}}
                \parens{v_{2} - \norm{v_{2}}_{2}e_{2}}.
\end{align}
%
From here, we let
%
\begin{align}
    \widehat{P}_{1} &= I_{n+1} - 2u_{1}u_{1}^{*} \nonumber\\
    \widehat{P}_{2} &= I_{n+1} - 2u_{2}u_{2}^{*} \nonumber\\
    \widehat{P} &= \widehat{P}_{1}\widehat{P}_{2} \nonumber\\
        &= I_{n+1} - 2\begin{bmatrix} u_{1} & u_{2} \end{bmatrix}
                \begin{bmatrix} u_{1} & u_{2} \end{bmatrix}^{*} \nonumber\\
    P &= \begin{bmatrix} I_{n} & 0 \\ 0 & \widehat{P} \end{bmatrix},
\end{align}
%
so that
%
\begin{equation}
    \label{eq:CV_D_1D_2n_E_F2_L_even}
    \begin{bmatrix} UWD \\ \overline{A} \end{bmatrix}D_{s}^{-1}G\Pi P
        = \begin{bmatrix}
            \gamma_{2} & \\
            & \gamma_{3} & \\
            & & \gamma_{4} & \\
            &   &   & \ddots \\
            & & & & \gamma_{n} & \\
            & & & & & \gamma_{n+1} & \\
            \alpha_{2} & 0 & \alpha_{4} & \cdots & \alpha_{n} & 0 &
                \norm{v_{1}}_{2} \\
            0 & \alpha_{3} & 0 & \cdots & 0 & \alpha_{n+1} &
                0 & \norm{v_{2}}_{2} \\
            \end{bmatrix}.
\end{equation}

When $n$ is odd, we let
%
\begin{equation}
    \overline{A} = C_{0}^{-1}A
\end{equation}
%
so that
%
\begin{equation}
    \label{eq:CV_D_1D_2n_E_F2_L_odd}
    \begin{bmatrix} UWD \\ \overline{A} \end{bmatrix}D_{s}^{-1}G\Pi
        = \begin{bmatrix}
            \gamma_{2} & \\
            & \gamma_{3} & \\
            & & \gamma_{4} & \\
            &   &   & \ddots \\
            & & & & \gamma_{n} & \\
            & & & & & \gamma_{n+1} & \\
            & \alpha_{3} & & \cdots & \alpha_{n} &
                & \alpha_{n+2} & & \cdots & & \alpha_{2n+1} & \alpha_{1} \\
            \alpha_{2} & & \alpha_{4} & \cdots & & \alpha_{n+1} & &
                \alpha_{n+3} & \cdots & \alpha_{2n} & & \\
            \end{bmatrix}.
\end{equation}
%
We set
%
\begin{align}
    v_{1}^{*} &= \begin{bmatrix} \alpha_{n+2} & 0 & \cdots &
                0 & \alpha_{2n+1} & \alpha_{1} \end{bmatrix} \nonumber\\
    v_{2}^{*} &= \begin{bmatrix} 0 & \alpha_{n+3} & 0 & \cdots & 0 &
                \alpha_{2n} & 0 & 0 \end{bmatrix}
\end{align}
%
and proceed as before.



\section{1D \CV{} Derivative Matrix: $2n+1$ Columns with Endpoint Interpolation;
    Second Factorization}
\label{sec:CV_D_1D_2n_E_F1}

In this section we compute the LQ factorization from last section using
the second factorization of the $V'$;
the endpoint function values are included as before
and begin by assuming $n$ is even.

We know that $C^{-1}V'D_{s}^{-1}GH$ essentially reduces our matrix to
lower triangular form, and the circular downshift permutation $\Pi$ 
puts the significant portion into lower triangular form.
Thus, after these applications, we arrive at the matrix
%
\begin{align}
    &\begin{bmatrix} \widehat{D}WD \\ \overline{A} \end{bmatrix}D_{s}^{-1}GH\Pi
        = \nonumber\\
    &\begin{bmatrix}
            * & \\
            * & * & \\
            * & * & * & \\
            \vdots &   &   & \ddots \\
            * & * & * & \cdots & * & \\
            * & * & * & \cdots & * & * & \\
            \alpha_{2} & & \alpha_{4} & \cdots & \alpha_{n} & &
                \alpha_{n+2} & & \cdots & \alpha_{2n} & & \\
            & \alpha_{3} & & \cdots & & \alpha_{n+1} &
                & \alpha_{n+3} & \cdots & & \alpha_{2n+1} & \alpha_{1} \\
            \end{bmatrix}.
\end{align}
%
The last two rows can be computed quickly by evaluating
$\overline{A}D_{s}^{-1}GH\Pi$.

All we have left is to compute 2 Householder reflectors, setting
%
\begin{align}
    v_{1}^{*} &= \begin{bmatrix} \alpha_{n+2} & 0 & \alpha_{n+4} & \cdots &
                \alpha_{2n} & 0 & 0 \end{bmatrix} \nonumber\\
    v_{2}^{*} &= \begin{bmatrix} 0 & \alpha_{n+3} & 0 & \cdots &
                0 & \alpha_{2n+1} & \alpha_{1} \end{bmatrix} \nonumber\\
    u_{1} &= \frac{1}{\norm{v_{1} - \norm{v_{1}}_{2}e_{1}}_{2}}
                \parens{v_{1} - \norm{v_{1}}_{2}e_{1}} \nonumber\\
    u_{2} &= \frac{1}{\norm{v_{2} - \norm{v_{2}}_{2}e_{2}}_{2}}
                \parens{v_{2} - \norm{v_{2}}_{2}e_{2}}.
\end{align}
%
From here, we let
%
\begin{align}
    \widehat{P}_{1} &= I_{n+1} - 2u_{1}u_{1}^{*} \nonumber\\
    \widehat{P}_{2} &= I_{n+1} - 2u_{2}u_{2}^{*} \nonumber\\
    \widehat{P} &= \widehat{P}_{1}\widehat{P}_{2} \nonumber\\
        &= I_{n+1} - 2\begin{bmatrix} u_{1} & u_{2} \end{bmatrix}
                \begin{bmatrix} u_{1} & u_{2} \end{bmatrix}^{*} \nonumber\\
    P &= \begin{bmatrix} I_{n} & 0 \\ 0 & \widehat{P} \end{bmatrix}.
\end{align}

\noindent
Putting this together, we have
%
\begin{equation}
    \label{eq:CV_D_1D_2n_E_F1_L_even}
    \begin{bmatrix} \widehat{D}WD \\ \overline{A}\end{bmatrix}D_{s}^{-1}GH\Pi P
        = \begin{bmatrix}
            * & \\
            * & * & \\
            * & * & * & \\
            \vdots &   &   & \ddots \\
            * & * & * & \cdots & * & \\
            * & * & * & \cdots & * & * & \\
            \alpha_{2} & 0 & \alpha_{4} & \cdots & \alpha_{n} & 0 &
                \norm{v_{1}}_{2} \\
            0 & \alpha_{3} & 0 & \cdots & 0 & \alpha_{n+1} &
                0 & \norm{v_{2}}_{2} \\
            \end{bmatrix},
\end{equation}
%
where we have only kept the nonzero terms. The inverse of this triangular
matrix can be computed quickly.

When $n$ is odd, we let
%
\begin{equation}
    \overline{A} = C_{0}^{-1}A.
\end{equation}
%
We then find
%
\begin{align}
    &\begin{bmatrix} \widehat{D}WD \\ \overline{A} \end{bmatrix}D_{s}^{-1}GH\Pi
        = \nonumber\\
    &\begin{bmatrix}
            * & \\
            * & * & \\
            * & * & * & \\
            \vdots &   &   & \ddots \\
            * & * & * & \cdots & * & \\
            * & * & * & \cdots & * & * & \\
            & \alpha_{3} & & \cdots & \alpha_{n} &
                & \alpha_{n+2} & & \cdots & & \alpha_{2n+1} & \alpha_{1} \\
            \alpha_{2} & & \alpha_{4} & \cdots & & \alpha_{n+1} & &
                \alpha_{n+3} & \cdots & \alpha_{2n} & & \\
            \end{bmatrix}.
    \label{eq:CV_D_1D_2n_E_F1_L_odd}
\end{align}
%
Setting
%
\begin{align}
    v_{1}^{*} &= \begin{bmatrix} \alpha_{n+2} & 0 & \cdots &
                0 & \alpha_{2n+1} & \alpha_{1} \end{bmatrix} \nonumber\\
    v_{2}^{*} &= \begin{bmatrix} 0 & \alpha_{n+3} & 0 & \cdots & 0 &
                \alpha_{2n} & 0 & 0 \end{bmatrix},
\end{align}
%
we proceed as before to arrive at the LQ factorization.



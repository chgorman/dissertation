\section{1D \CV{} Interpolation Matrix: $2Ln+1$ Columns}
\label{sec:CV_1D_2Ln}

The previous interpolation patterns showed us a standard way to 
compute the pseudoinverse: compute the necessary LQ factorization
using structured rotations.
This is always possible but becomes difficult as the interpolation
degree becomes larger; previously, we have always computed
the full $Q$ matrix, but the $Q$ factor became more involved when
using $3n+1$ and $4n+1$ columns.
If we wish to have a method that works for MSN interpolation up to degree
$2Ln$, then we will need to use something else: the normal equations.
While it is well-known the normal equations can increase
the difficulty of solving linear systems because it results
in a squaring of the condition number~\cite{gvl4,HighamASNA},
this is not an issue here because the system is well structured and
we are computing the factorization exactly (by hand).

After performing an IDCT, we see
%
\begin{equation}
    \parens{WD_{s}^{-1}}^{+} = D_{s}^{-1}W^{*}
        \parens{WD_{s}^{-2}W^{*}}^{-1},
\end{equation}
%
where we are assuming $W$ has $2Ln+1$ columns.
$W$ has the following form:
%
\begin{align}
    W &= \begin{bmatrix}
        I & 0 & -\Lambda & 0 & \Lambda & \cdots & 0 & E \end{bmatrix},
    %\quad \mod(L,4) = 0 \nonumber\\
    \quad L \mod 4 = 0 \nonumber\\
    W &= \begin{bmatrix}
        I & 0 & -\Lambda & 0 & \Lambda & \cdots & 0 & -E \end{bmatrix},
    %\quad \mod(L,4) = 2.
    \quad L \mod 4 = 2.
\end{align}
%
We see
%
\begin{equation}
    WD_{s}^{-2}W^{*} = Y^{2},
\end{equation}
%
where $Y$ is a diagonal matrix with
%
\begin{align}
    y_{1}^{2} &= 1^{-2s} + \brackets{2n+1}^{-2s} + \brackets{4n+1}^{-2s}
        + \cdots + \brackets{2Ln+1}^{-2s}
        \nonumber\\
    y_{k}^{2} &= k^{-2s} + \brackets{2n+2-k}^{-2s} + \brackets{2n+k}^{-2s}
        + \brackets{4n+2-k}^{-2s} + \brackets{4n+k}^{-2s} \nonumber\\
        &\quad+ \cdots
        + \brackets{2(L-1)n+2-k}^{-2s} + \brackets{2(L-1)n+k}^{-2s}
        + \brackets{2Ln+2-k}^{-2s} \nonumber\\
    &\quad k\in\braces{2,\cdots,n}.
\end{align}

\noindent
If there is concern about roundoff error in these calculations,
the values could be first sorted into increasing order before
being added or compensated summation could be used;
see~\cite[Chapter 4]{HighamASNA}.
In this case, we see
%
\begin{equation}
    \parens{WD_{s}^{-1}}^{+} = D_{s}^{-1}W^{*}Y^{-2},
\end{equation}
%
and all of these operations can be performed quickly.
The solution coefficients are given by
%
\begin{equation}
    a = D_{s}^{-2}W^{*}Y^{-2}\hat{f}.
\end{equation}


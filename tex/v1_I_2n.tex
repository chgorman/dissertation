\section{1D \CV{} Interpolation Matrix: $2n+1$ Columns}
\label{sec:CV_1D_2n}

We take $V$ to be our \CV{} matrix with $2n+1$ columns.
Recalling that we need to compute the minimum norm solution of
%
\begin{equation}
    VD_{s}^{-1}x = f.
\end{equation}
%
$D_{s}$ is a diagonal scaling matrix which we use
to bound the $s$th derivative; it was defined in
Eq.~\eqref{eq:kar_ds_def}.
From here, we see
%
\begin{align}
    C^{-1}VD_{s}^{-1} &= WD_{s}^{-1} \nonumber\\
        &= \begin{bmatrix}
            1^{-s} & & & & & & & & & -\parens{2n+1}^{-s} \\
            & 2^{-s} & & & & & & & -\parens{2n}^{-s} & \\
            & & \ddots & & & & & \iddots & & \\
            & & & & n^{-s} & 0 & -\parens{n+2}^{-s} & & & \\
        \end{bmatrix}.
\end{align}

We need to convert this matrix into a lower triangular matrix via rotations,
and we use Givens rotations to zero the nonzero components.
Specifically, we require the Givens rotation $G_{k}$ to have
%
\begin{equation}
    G_{k}\parens{\brackets{k,2n+2-k},\brackets{k,2n+2-k}} = 
        \begin{bmatrix} c_{k} & s_{k} \\ -s_{k} & c_{k} \end{bmatrix}.
\end{equation}
%
In practice, we can compute 
%
\begin{align}
    \tau_{k} &= \parens{\frac{k}{2n+2-k}}^{s} \nonumber\\
    c_{k} &= \frac{1}{\sqrt{1 + \tau_{k}^{2}}} \nonumber\\
    s_{k} &= \tau_{k}c_{k}
    \label{eq:v1_I_2n_Givens_cs_def}
\end{align}
%
stably using Alg.~\ref{alg:givens}.
All of these Givens rotations are
disjoint and we set $G = G_{1}G_{2}\cdots G_{n}$, with
%
\begin{equation}
    G = \begin{bmatrix}
        c_{1} &       &        &       &      &       &        &       & s_{1}\\
              & c_{2} &        &       &      &       &        & s_{2} &      \\
              &       & \ddots &       &      &       & \iddots&       &      \\
              &       &        & c_{n} &      & s_{n} &        &       &      \\
              &       &        &       &    1 &       &        &       &      \\
              &       &        & -s_{n}&      & c_{n} &        &       &      \\
              &       & \iddots&       &      &       & \ddots &       &      \\
              &-s_{2} &        &       &      &       &        & c_{2} &      \\
        -s_{1}&       &        &       &      &       &        &       & c_{1}\\
        \end{bmatrix}.
    \label{eq:vand_I_Givens}
\end{equation}

Then, we find
%
\begin{equation}
    WD_{s}^{-1}G = \begin{bmatrix} Y & 0 \end{bmatrix},
\end{equation}
%
where
%
\begin{align}
    Y &= \diag\parens{y_{1},\cdots,y_{n}} \nonumber\\
    y_{k} &= k^{-s}\sqrt{1 + \parens{\frac{k}{2n+2-k}}^{2s}} \nonumber\\
    &= \frac{1}{c_{k}k^{s}}.
    \label{eq:v1_I_2n_Givens_y_def}
\end{align}
%
This is the desired LQ factorization, and we can now use
Alg.~\ref{alg:general_lq_solution} to solve this system quickly.
As we can see, we are making small changes to the DCT coefficients.



\section{Endpoint Interpolation}
\label{sec:CV_1D_Endpoints}

While most of the factorizations involve the Chebyshev nodes, we will
at times include the endpoints $\pm1$. If we want to interpolate
function values at $f(-1) = f_{0}$ and $f(1) = f_{n+1}$, then the
coefficients $a$ must satisfy
%
\begin{align}
    Aa &= \begin{bmatrix} 1 & -1 & 1 & -1 & \cdots & -1 & 1 \\
            1 & 1 & 1 & 1 & \cdots & 1 & 1 \end{bmatrix}
            \begin{bmatrix} a_{0} \\ a_{1} \\ a_{2} \\ a_{3} \\ \vdots \\
                a_{2n-1} \\ a_{2n} \end{bmatrix} \nonumber\\
        &= \begin{bmatrix} f_{0} \\ f_{n+1} \end{bmatrix}.
        \label{eq:vand_pm1}
\end{align}
%
We define
%
\begin{align}
    C_{0} &= \begin{bmatrix} 1 & -1 \\ 1 & 1 \end{bmatrix} \nonumber\\
    C_{1} &= \begin{bmatrix} -1 & 1 \\ 1 & 1 \end{bmatrix}.
    \label{eq:vand_pm1_C}
\end{align}
%
With these definitions, we see
%
\begin{align}
    C_{0}^{-1}A &= \begin{bmatrix} 1 & 0 & 1 & 0 & \cdots & 0 & 1 \\
            0 & 1 & 0 & 1 & \cdots & 1 & 0 \end{bmatrix} \nonumber\\
    C_{1}^{-1}A &= \begin{bmatrix} 0 & 1 & 0 & 1 & \cdots & 1 & 0 \\
            1 & 0 & 1 & 0 & \cdots & 0 & 1 \end{bmatrix}.
\end{align}
%
We will prefer one of these matrices ($C_{0}$ or $C_{1}$) over the
other depending on the context. The distinction depends on whether
$n$ is even or odd.




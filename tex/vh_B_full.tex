\section{2D Full Birkhoff Interpolation Problem}
\label{sec:CV_2D_full_birkhoff}

We now look at the full Birkhoff interpolation problem in 2D:
%
\begin{align}
    f(z_{i}^{n},z_{j}^{n}) &= p(z_{i}^{n},z_{j}^{n}),
        \quad i,j\in\braces{1,\cdots,n} \nonumber\\
    f_{x}(z_{i}^{n},z_{j}^{n}) &= p_{x}(z_{i}^{n},z_{j}^{n}),
        \quad i,j\in\braces{1,\cdots,n} \nonumber\\
    f_{y}(z_{i}^{n},z_{j}^{n}) &= p_{y}(z_{i}^{n},z_{j}^{n}),
        \quad i,j\in\braces{1,\cdots,n}.
\end{align}
%
That is, we interpolate all function and first derivative values
on the $n\times n$ Chebyshev tensor grid.

After some of the standard matrix preprocessing, we must compute
the LQ factorization of the following matrix:
%
\begin{equation}
    \begin{bmatrix}
        WD_{s}^{-1} \otimes WD_{s}^{-1} \\
        EUWD_{s}^{-1} \otimes WD_{s}^{-1} \\
        WD_{s}^{-1} \otimes EUWD_{s}^{-1} \\
    \end{bmatrix}.
\end{equation}
%
We assume that we are interpolating up to degree $N$ in both $x$
and $y$ (not specified except that $N\ge2n$).
If $G$ is the structured rotation matrix so that
%
\begin{equation}
    WD_{s}^{-1}G = \begin{bmatrix} Y & 0 \end{bmatrix},
\end{equation}
%
then
%
\begin{equation}
    \begin{bmatrix}
        WD_{s}^{-1} \otimes WD_{s}^{-1} \\
        EUWD_{s}^{-1} \otimes WD_{s}^{-1} \\
        WD_{s}^{-1} \otimes EUWD_{s}^{-1} \\
    \end{bmatrix}
    \parens{G\otimes G} = 
    \begin{bmatrix}
        \begin{bmatrix} Y & 0 \end{bmatrix} \otimes
            \begin{bmatrix} Y & 0 \end{bmatrix} \\
        EUWD_{s}^{-1}G \otimes \begin{bmatrix} Y & 0 \end{bmatrix} \\
        \begin{bmatrix} Y & 0 \end{bmatrix} \otimes EUWD_{s}^{-1}G \\
    \end{bmatrix}.
\end{equation}
%
We now take $H$ to be the rotation matrix so that
%
\begin{align}
    \begin{bmatrix} Y & 0 \end{bmatrix}H &= \begin{bmatrix} Y & 0 \end{bmatrix}
        \nonumber\\
    \parens{EUWD_{s}^{-1}G}H &= \begin{bmatrix} A & B & 0 \end{bmatrix}.
\end{align}
%
This rotation matrix $H$ comes up in the full Birkhoff problem in 1D
and will contain some Givens rotations when $N>2n+1$.
Using this, we see

\begin{equation}
    \begin{bmatrix}
        \begin{bmatrix} Y & 0 \end{bmatrix} \otimes
            \begin{bmatrix} Y & 0 \end{bmatrix} \\
        EUWD_{s}^{-1}G \otimes \begin{bmatrix} Y & 0 \end{bmatrix} \\
        \begin{bmatrix} Y & 0 \end{bmatrix} \otimes EUWD_{s}^{-1}G \\
    \end{bmatrix}\parens{H\otimes H} = 
    \begin{bmatrix}
        \begin{bmatrix} Y & 0 & 0 \end{bmatrix} \otimes
            \begin{bmatrix} Y & 0 & 0 \end{bmatrix} \\
        \begin{bmatrix} A & B & 0 \end{bmatrix} \otimes
            \begin{bmatrix} Y & 0 & 0 \end{bmatrix} \\
        \begin{bmatrix} Y & 0 & 0 \end{bmatrix} \otimes
            \begin{bmatrix} A & B & 0 \end{bmatrix} \\
    \end{bmatrix}.
\end{equation}

At this point, we can compute the solution, once we use the following
index sets:
%
\begin{align}
    I_{1} &= \bigcup_{k=1}^{2n} \brackets{(k-1)N + 1: (k-1)N + n} \nonumber\\
    I_{2} &= \bigcup_{k=1}^{n} \brackets{(k-1)N + (n+1): (k-1)N + 2n}.
\end{align}
%
In this case, we need to solve
%
\begin{align}
    \parens{\begin{bmatrix} Y & 0 \\ A & B \end{bmatrix} \otimes Y}y(I_{1})
        &= \begin{bmatrix} \hat{f} \\ \parens{U\otimes I}\hat{f}_{y}
        \end{bmatrix} \nonumber\\
    \parens{Y\otimes B}y(I_{2}) &= \parens{I\otimes U}\hat{f}_{x}.
\end{align}
%
The rest of the entries of $y$ are initialized to zero.
Here, we remember that we compute hat's by the 2D IDCT.
From here, we can easily solve for the coefficients:
%
\begin{equation}
    a = \overline{D}_{s}^{-1} \overline{G} \overline{H}y.
\end{equation}

This problem is useful because of how difficult it would
be to solve in another setting.
If we wish to keep the tensor structure with each matrix
having the same number of columns $N$, then we have $3n^{2}$ linear
constraints and $N^{2}$ total columns.
It is not possible for $3n^{2} = N^{2}$ to hold for all $n$ and $N$
when we are forced to have a square system.
The MSN method does not care about this restriction and so we can
choose $N$ freely, so long as it is large enough that the structured
rotations go through.
While fast algorithms may exist without the tensor structure,
it is not clear how this could be determined or exploited.



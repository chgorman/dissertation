\section{\CV{} Matrices in Higher Dimensions}
\label{sec:CV_higher}

One difficulty going from 1D to 2D interpolation, as noted in
Chapter~\ref{chap:intro}, is how
to chose the correct set of basis functions~\cite{gasca2000polynomial}.
Because we are not constrained by matching the number
of interpolation requirements with basis functions, our work is easier.
We let $\overline{V} = V^{y} \otimes V^{x}$ be our 2D \CV{}
matrix. Here, $V^{x}\in\R^{n\times(2n+1)}$
is our 1D \CV{} matrix in $x$
and $V^{y}\in\R^{m\times(2m+1)}$ is our 1D \CV{} matrix in $y$.
We similarly define $\overline{C} = C^{y}\otimes C^{x}$ and
$\overline{G} = G^{y}\otimes G^{x}$.
Due to the nature of the Kronecker product, many of the properties
from 1D interpolation are kept but the method for computing them
must be modified.

For interpolation on the Chebyshev nodes in 2D, we want to solve
the following system:
%
\begin{equation}
    \min_{\overline{V}a=f}\norm{\overline{D}_{s}a}_{2}.
\end{equation}
%
The solution will be of the form
%
\begin{equation}
    p(x,y) = \sum_{k=0}^{2n}\sum_{\ell=0}^{2m} a_{k,\ell}T_{k}(x)T_{\ell}(y).
\end{equation}
%
Because of how $\overline{V}$ is defined, the coefficients are ordered
%
\begin{equation}
    a^{*} = \begin{bmatrix} a_{0,0} & a_{1,0} & \cdots &
                        a_{2n,0} & a_{0,1} & a_{1,1} & \cdots &
                        a_{2n,1} & \cdots &
                        a_{0,2m} & a_{1,2m} & \cdots & a_{2n,2m}
        \end{bmatrix}.
\end{equation}
%
Furthermore, our interpolation points are ordered
%
\begin{equation}
    \parens{x_{1},y_{1}},\parens{x_{2},y_{1}},\cdots,\parens{x_{n},y_{1}},
    \parens{x_{1},y_{2}},\cdots.
\end{equation}
%
Thus, if $x = \overline{D}_{s}a$ where
then we need to compute
%
\begin{equation}
    \min_{\overline{V}\overline{D}_{s}^{-1}x = f}\norm{x}_{2}.
\end{equation}
%
Our work reduces to computing the LQ factorization of
$\overline{V}\overline{D}_{s}^{-1}$.

Now, we know
%
\begin{equation}
    \overline{C}^{-1}\overline{V}\overline{D}_{s}^{-1}\overline{G}
        = \begin{bmatrix} Y^{y} & 0 \end{bmatrix}
            \otimes \begin{bmatrix} Y^{x} & 0 \end{bmatrix}
\end{equation}
%
This is \emph{not} the $LQ$ factorization, but it is after
a permutation of the columns.

We now assume $m=n$.
Let
%
\begin{equation}
    I = \bigcup_{k=1}^{n}
        \brackets{\parens{k-1}\parens{2n+1}+1\!:\!\parens{k-1}\parens{2n+1}+n},
\end{equation}
%
so that
%
\begin{equation}
    \parens{Y^{y}\otimes Y^{x}}y\parens{I} = f,
\end{equation}
%
where we initialize the remaining elements of $y$ to zero.
At this point, we compute
%
\begin{equation}
    a = \overline{D}_{s}^{-1}\overline{G}y.
\end{equation}
%
Similar results can be extended to higher dimensions.
While it is not necessary to require $m=n$, we will in the future
because it simplifies matters.



